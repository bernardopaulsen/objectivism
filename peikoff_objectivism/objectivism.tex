% Title      : Objectivism
% Description: Book 'Objectivism: the Philosophy of Ayn Rand (Notes)'.
% Author     : Bernardo Paulsen
% Version    : 0.6.0

\documentclass{book}

\usepackage[utf8]{inputenc}
\usepackage[a5paper]{geometry}
\usepackage{amsthm}
\usepackage{comment}
\usepackage[nottoc,numbib]{tocbibind} % bibliography in table of contents
\usepackage{natbib}
\bibliographystyle{apalike}
\usepackage{todonotes}
\usepackage{blindtext}
\usepackage{hyperref}

\hypersetup{
    colorlinks=true,
    linkcolor=blue,
    filecolor=magenta,      
    urlcolor=cyan,
    pdftitle={Objectivism: the Philosophy of Ayn Rand (Notes)},
    pdfpagemode=FullScreen,
    }

\newtheorem{axiom}{Axiom}[section]
\newtheorem*{example}{Example}
\newtheorem{corollary}{Corollary}[section]
\newtheorem{definition}{Definition}[section]
\newtheorem*{remark}{Remark}
\newtheorem{theorem}{Theorem}[section]

\title{Objectivism: the Philosophy of Ayn Rand \\
    (Notes)}
\author{Bernardo Paulsen}

\begin{document}

\maketitle
\tableofcontents

\frontmatter

\chapter*{Dedication}
\addcontentsline{toc}{chapter}{Dedication}

    To my loved ones, and to every human who strives for the good.
\chapter*{Preface}
\addcontentsline{toc}{chapter}{Preface}

    This book is based on \cite{peikoff1993}. Peikoff's book develops the philosophy of Objectivism systematically. Nevertheless, I believe it would be interesting to see it written in a more straightforward manner. That is what I try to accomplish in this book.
    
    %Now, a disclaimer: I'm (extremely) new to philosophy, and I have never studied philosophy formally - only by reading books by my own. I have written this book beacuse Objectivism - especially as a full philosophical system - helped me understand the interconnections and interdependences between ideas as no other philosophy ever did. I do not expect the ideas on this book to be perfectly developed. Also I do not expect to forever hold the opinions that derive from them. However, I do expect to continue to believe in what I consider the main ideas presented in this book - reason and liberty.
    

    I hope you enjoy.
\chapter{Glossary}

     We begin by defining ``true'' (Definition \ref{def:true}) and ``self-evidence'' (Definition \ref{def:self_evidence}).

        \begin{definition}[True]
        \label{def:true}
            In accordance with reality.
        \end{definition}

        \begin{definition}[Self-Evidence]
        \label{def:self_evidence}
            A proposition that is known to be true by sense perception, without the need of proof.
        \end{definition}

    Now, we can define ``axiom'', ``corollary'', ``theorem'' and ``proof'' (Definitions \ref{def:axiom}, \ref{def:corollary}, \ref{def:theorem} and \ref{def:proof}).

        \begin{definition}[Axiom]
        \label{def:axiom}
            A self-evidence.
        \end{definition}

        \begin{definition}[Corollary]
        \label{def:corollary}
            A self-evidence that follows from another self-evidence.
        \end{definition}

        \begin{definition}[Theorem]
        \label{def:theorem}
            A proposition that is not self-evident but it proved by a chain of reasoning; a truth established by means of accepted truths.
        \end{definition}

        \begin{definition}[Proof]
        \label{def:proof}
            Argument establishing the truth of a statement.
        \end{definition}

    This book is about philosophy. Below are the definitions of the five branches of philosohy the will be discussed: metaphysics, epistemology, ethics, politics and aesthetics (Definitions \ref{def:metaphysics}, \ref{def:epistemology}, \ref{def:ethics}, \ref{def:politics} and \ref{def:aesthetics}) - which are discussed in Parts \ref{part:metaphysics}, \ref{part:epistemology}, \ref{part:ethics}, \ref{part:politics} and \ref{part:aesthetics} respectively.

        \begin{definition}[Metaphysics]
        \label{def:metaphysics}
            The branch of philosophy that deals with the first principles of things.
        \end{definition}

        \begin{definition}[Epistemoly]
        \label{def:epistemology}
            The branch of phislosophy that deals with knowledge, especially with regard to its methods, validity, and scope.
        \end{definition}

        \begin{remark}
            Knowledge is a thing, therefore metaphysics deals with first principles of knowledge.
        \end{remark}

        \begin{definition}[Ethics]
        \label{def:ethics}
            The branch of knowledge that deals with moral principles.
        \end{definition}

        \begin{definition}[Politics]
        \label{def:politics}
            The branck of knowledge that deals with government and the state.
        \end{definition}

        \begin{definition}[Aesthetics]
        \label{def:aesthetics}
            The branch of kanowledge that deals with the principles of beauty and art.
        \end{definition}

    New concepts will be defined throughout the text.

\mainmatter

\part{Metaphysics}
\label{part:metaphysics}

\input{mainmatter/1}
\input{mainmatter/2}

\part{Epistemology}
\label{part:epistemology}

\input{mainmatter/3}
\input{mainmatter/4}
\input{mainmatter/5}
\input{mainmatter/6}
\input{mainmatter/7}
\chapter{Virtue}

    Above we argued rationality is the primary virtue. Now, we turn to the derivative virtues of rationality.

    \section{Independence as a Primary Orientation to Reality, not to Other Men}

        \begin{definition}[Independence]
        \label{def:independence}
            One's accpetance of the responsibility of forming one's own judgments and of living by the work of one's own mind.
        \end{definition}

        Rationality is the primary virtue, and reason is a faculty of the individual. Therefore, in order to be rational, one must think independently (a pleonasm). This means not accepting thoughts or jugdments of others, but to form one's own. This does not mean never thinking like other people or accepting ideas created by other, but it does mean that this acceptance must be rational, the thoughts must be integrated by an act of reason - thinking.

    \section{Integrity as Loyalty to Rational Principles}

        \begin{definition}[Integrity]
        \label{def:integrity}
            Loyalty in action to one's convictions and values.
        \end{definition}

        Independence is forming one's own ideas. Integrety is following one's own ideas, acting rationally. Reality, to be mastered, must be known, but it also requires that man acts on his knowledge. There is no use for knowledge if it is not to be acted upon.

    \section{Honesty as the Rejection of Unreality}

        

    \section{Justice as Rationality in the Evaluation of Men}

        

    \section{Productiveness as the Adjustment of Nature to Man}

        \begin{definition}[Productiveness]
        \label{def:productiveness}
            The process of creating material values.
        \end{definition}

        Productiveness is a necessity of human survival. Just as there is not such a thing as ``too much'' knowledge, there is not such a thing as ``too much'' wealth. Intelectually, every discovery enhances human life; existentially, every material achievement also enhances human life.

        Productiveness is the creation of \textit{material} values, not because knowledge is important, but because it is important exclactly because it is useful in creating material values.

    \section{Pride as Moral Ambitiousness}

        \begin{definition}[Pride]
            A feeling of deep pleasure or satisfaction derived from one's own achievements, the achievements of those with whom one is closely associated, or from qualities or possessions that are widely admired.
        \end{definition}

        \begin{definition}[Self-esteem]
            Confidence in one's own worth or abilities; self-respect.
        \end{definition}

        Pride and self-esteem are results of virtue, of moral action.


    \section{The Initiation of Physical Force as Evil}

        Initiation of physical force is the destruction of rationality, and therefore of every other virtue and of every value. To force a man to do something is to deny him the possibility of acting on reason, thus turning any virtuous action (value achievement) impossible.
        
        A (rational) victim of force still thinks what he thinks, as force cannot change the content of one's mind. Thus the victim of force acts \textit{against} his judgement, and his cognition becomes useless. As a rational mind acts virtuously, to act against its judgement is to act perversely, to act against one's own life. Initiating force, when it does not kill the victim, kills his capacity to live.

        Coercion places the individual in an impossible metaphysical condition. If he does not act in accordance with his conlusions, he is doomed by reality. If he does, he is doomed by the forcer.

        Finally, one important thing to notice is that only initiation - not retaliation - of physical force is evil. While initiation of physical force destroys rationality, retaliation protects it. It cannot creates values, however - it can only protect them. 
\chapter{Happiness}

    The existential reward of virtue is life, and the emotional reward is happiness. Happiness is the \textit{only} moral purpuse of man's life.

    \section{Virtue as Practical}

        \begin{definition}[Practical]
        \label{def:practical}
            \begin{enumerate}
                \item Of or concerned with the actual doing or use of something rather than with theory and ideas.
                \item (Of an idea, plan, or method) likely to succeed or be effective in real circumstances; feasible.
            \end{enumerate}
        \end{definition}

        The moral man's concept of the good - life - is his fundamental standard of practicality.

    \section{Happiness as the Normal Condition of Man}

        There are two opposite emotions - the first follows the achievement of value,
            the other from its loss - they are joy and suffering (Definitions~\ref{def:joy} and \ref{def:suffering}).
        Happiness (Definition \ref{def:happiness}) follows from joy.

        \begin{definition}[Joy]
        \label{def:joy}
            Emotion that follows from the achievement of values.
        \end{definition}

        \begin{definition}[Suffering]
        \label{def:suffering}
            Emotion that follows from the loss of values.
        \end{definition}

        \begin{definition}[Happiness]
        \label{def:happiness}
            A state of noncontradictory joy.
        \end{definition}

        To feel happiness, it is necessary to achieve noncontradictory - rational - values. Therefore the moral, the practical and the happy are the same. To feel happiness is to feel a metahphysical pleasure: the pleasure of being in the right relationship to existence. Happiness is the purpose of ethics, it is an end in itself. A rational man must fight for and expect happiness.

        While joy and suffering follow from virtue, which pressuposes evaluation, from evaluation itself love and fear (Definitions \ref{def:love} and \ref{def:fear}) follow.

        \begin{definition}[Love]
        \label{def:love}
            Emotion that follows from the appreciation of values.
        \end{definition}

        \begin{definition}[Fear]
        \label{def:fear}
            Emotion that follows from the depreciation of disvalues.
        \end{definition}

    \section{Sex}

        Here we shall consider sex as in the life of a rational man.

        \begin{definition}[Like]
        \label{def:like}
            Find agreeable, enjoyable (Definition \ref{def:joy}), or satisfactory.
        \end{definition}

        \begin{definition}[Attraction]
        \label{def:attraction}
            The action or power of evoking interest, pleasure, or liking (Definition \ref{def:like}) for someone or something.
        \end{definition}

        \begin{definition}[Sexual]
        \label{def:sexual}
            Relating to the instincts, physiological processes,
                and activities connected with physical attraction (Definition \ref{def:attraction}]) or intimate physical contact between individuals.
        \end{definition}

        \begin{definition}[Sex]
        \label{def:sex}
             Sexual (Definition \ref{def:sexual}]) activity, including specifically sexual intercourse.
         \end{definition}

%        To celebrate something is to focus on it. Sex is a celebration of love. Love itself is the appreciation of values. Since the value of a man is his virtue and the latter is his volition, which is who he truly is, to celebrate self love is to celebrate oneself. To celebrate love towards the other is also to celebrate who the other truly is.
%
%        relating to the instincts, physiological processes, and activities connected with physical attraction or intimate physical contact between individuals.

        \subsection{Monogamy}

            \begin{definition}[Monogamy]
            \label{def:monogamy}
                The practice or state of having a sexual relationship with only one partner.
            \end{definition}

        \subsection{Orgy}

            \begin{definition}[Orgy]
            \label{def:orgy}
                A wild party, especially one involving excessive drinking and unrestrained sexual activity.
            \end{definition}

            \begin{definition}[Party]
            \label{def:party}
                A social gathering of invited guests, typically involving eating, drinking, and entertainment.
            \end{definition}

            \begin{definition}[Gathering]
            \label{def:gathering}
                An assembly or meeting, especially a social or festive one or one held for a specific purpose.
            \end{definition}

            \begin{definition}[Festive]
            \label{def:festive}
                Cheerful and jovially celebratory.
            \end{definition}

            \begin{definition}[Celebrate]
            \label{def:celebrate}
                Acknowledge (a significant or happy day or event) with a social gathering or enjoyable activity.
            \end{definition}

            \begin{definition}[Aknowledge]
                Accept or admit the existence or truth of.
            \end{definition}


\part{Politics}
\label{part:politics}

\chapter{Government}

    Politics, like ethics, is a normative branch of philosophy. Politics defines the principles of a proper social system, including the proper functions of government. Politics rests on ethics (and thus on metaphysics and spistemoloy); it is an application of ethics to social questions.

    What type of society conforms to or reflects the principles of morality? - this is the question asked by philosophicas politics. In Objectivism, the question is: what type of society conforms to the requirements of man's life? What makes virtues possible?

    \section{Individual Rights as Absolutes}

        The basic principle of politics is the principle of individual rights. Inidividual rights are the means of subordinating society to moral law.

        \begin{definition}[Right]
            A moral or legal entitlement to have or obtain something or to act in a certain way.
        \end{definition}

        If your society is to be moral (and therefore practical), it declares, you must begin by recognizing the moral requirements of man in a social context; i.e., you must define the sphere of sovereignty mandated for every individual by the laws of morality.

        The fundamental right is the right to life. Its major derivatives are the right to liberty, property, and the pursuit of happiness.

        The right to life means the right to sustain and protect one's life. It means the right to take all the actions required by the nature of a rational being for the preservation of his life. To sustain his life, man needs a method of survival—he must use his rational faculty to gain knowledge and choose values, then act to achieve his values. The right to liberty is the right to this method; it is the right to think and choose, then to act in accordance with one's judgment. To sustain his life, man needs to create the material means of his survival. The right to property is the right to this process; in Ayn Rand's definition, it is "the right to gain, to keep, to use and to dispose of material values." To sustain his life, man needs to be governed by a certain motivehis purpose must be his own welfare. The right to the pursuit of happiness is the right to this motive; it is the right to live for one's own sake and fulfillment.

        Since man is an integrated being of mind and body, every right entails every other; none is definable or possible apart from the rest.

        Man is a certain kind of living organism—which leads to his need of morality and to man s life being the moral standard—which leads to the right to act by the guidance of this standard, i.e., the right to life. Reason is man's basic means of survival—which leads to rationality being the primary virtue—which leads to the right to act according to one s judgment, i.e., the right to liberty. Unlike animals, man- does not survive by adjusting to the given—which leads to productiveness being a cardinal virtue—which leads to the right to keep, use, and dispose of the things one has produced, i.e., the right to property. Reason is an attribute of the indi- vidual, one that demands, as a condition of its function, un- breached allegiance to reality—which leads to the ethics of egoism—which leads to the right to the pursuit of happiness.

        All rights rest on the fact that man's life is the moral stan- dard. Rights are rights to the kinds of actions necessary for the preservation of human life.

        All rights rest on the ethics of egoism. Rights are an in- dividual's selfish possessions—his title to his life, his liberty, his property, the pursuit of his own happiness.

        By its nature, the concept of a "right" pertains, in Ayn Rand's words, "only to action—specifically, to freedom of ac- tion. It means freedom from physical compulsion, coercion or interference by other men."
        
        A man's rights impose no duties on others, but only a negative obligation: oth- ers may not properly violate his rights.

        "Individual rights," in short, is a redundancy, albeit a neces- sary one in today's intellectual chaos. Only the individual has rights.

        The rights of man, Ayn Rand holds, can be violated by one means only: by the initiation of physical force (including its indirect forms, such as fraud).

        Metaphysically, the individual is sovereign (he is a being of self-made soul). Ethically, he is obliged to live as a sovereign (as an independent egoist). Politically, therefore, he must be able to act as a sovereign.

    \section{Government as an Agency to Protect Rights}

    \section{Statism as the Politics of Unreason}
\input{mainmatter/11}
\input{mainmatter/12}

\backmatter

\bibliography{references}

\end{document}