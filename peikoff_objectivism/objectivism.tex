% Title      : Objectivism
% Description: Book 'Objectivism: the Philosophy of Ayn Rand (Notes)'.
% Author     : Bernardo Paulsen
% Version    : 0.6.0

\documentclass{book}

\usepackage[utf8]{inputenc}
\usepackage[a5paper, total={5in, 8in}]{geometry}
\usepackage{amsthm}
\usepackage{comment}
\usepackage[nottoc,numbib]{tocbibind} % bibliography in table of contents
\usepackage{natbib}
\bibliographystyle{apalike}
\usepackage{todonotes}
\usepackage{blindtext}
\usepackage{hyperref}

\hypersetup{
    colorlinks=true,
    linkcolor=blue,
    filecolor=magenta,      
    urlcolor=cyan,
    pdftitle={Objectivism: the Philosophy of Ayn Rand (Notes)},
    pdfpagemode=FullScreen,
    }

\newtheorem{axiom}{Axiom}[section]
\newtheorem*{example}{Example}
\newtheorem{corollary}{Corollary}[section]
\newtheorem{definition}{Definition}[section]
\newtheorem*{remark}{Remark}
\newtheorem{theorem}{Theorem}[section]

\title{Objectivism: the Philosophy of Ayn Rand \\
    (Notes)}
\author{Bernardo Paulsen}

\begin{document}

\maketitle
\tableofcontents

\frontmatter

\chapter*{Dedication}
\addcontentsline{toc}{chapter}{Dedication}

    To my loved ones, and to every human who strives for the good.
\chapter*{Preface}
\addcontentsline{toc}{chapter}{Preface}

    This book is based on \cite{peikoff1993}. Peikoff's book develops the philosophy of Objectivism systematically. Nevertheless, I believe it would be interesting to see it written in a more straightforward manner. That is what I try to accomplish in this book.
    
    %Now, a disclaimer: I'm (extremely) new to philosophy, and I have never studied philosophy formally - only by reading books by my own. I have written this book beacuse Objectivism - especially as a full philosophical system - helped me understand the interconnections and interdependences between ideas as no other philosophy ever did. I do not expect the ideas on this book to be perfectly developed. Also I do not expect to forever hold the opinions that derive from them. However, I do expect to continue to believe in what I consider the main ideas presented in this book - reason and liberty.
    

    I hope you enjoy.
\chapter{Glossary}

     We begin by defining ``true'' (Definition \ref{def:true}) and ``self-evidence'' (Definition \ref{def:self_evidence}).

        \begin{definition}[True]
        \label{def:true}
            In accordance with reality.
        \end{definition}

        \begin{definition}[Self-Evidence]
        \label{def:self_evidence}
            A proposition that is known to be true by sense perception, without the need of proof.
        \end{definition}

    Now, we can define ``axiom'', ``corollary'', ``theorem'' and ``proof'' (Definitions \ref{def:axiom}, \ref{def:corollary}, \ref{def:theorem} and \ref{def:proof}).

        \begin{definition}[Axiom]
        \label{def:axiom}
            A self-evidence.
        \end{definition}

        \begin{definition}[Corollary]
        \label{def:corollary}
            A self-evidence that follows from another self-evidence.
        \end{definition}

        \begin{definition}[Theorem]
        \label{def:theorem}
            A proposition that is not self-evident but it proved by a chain of reasoning; a truth established by means of accepted truths.
        \end{definition}

        \begin{definition}[Proof]
        \label{def:proof}
            Argument establishing the truth of a statement.
        \end{definition}

    This book is about philosophy. Below are the definitions of the five branches of philosohy the will be discussed: metaphysics, epistemology, ethics, politics and aesthetics (Definitions \ref{def:metaphysics}, \ref{def:epistemology}, \ref{def:ethics}, \ref{def:politics} and \ref{def:aesthetics}) - which are discussed in Parts \ref{part:metaphysics}, \ref{part:epistemology}, \ref{part:ethics}, \ref{part:politics} and \ref{part:aesthetics} respectively.

        \begin{definition}[Metaphysics]
        \label{def:metaphysics}
            The branch of philosophy that deals with the first principles of things.
        \end{definition}

        \begin{definition}[Epistemoly]
        \label{def:epistemology}
            The branch of phislosophy that deals with knowledge, especially with regard to its methods, validity, and scope.
        \end{definition}

        \begin{remark}
            Knowledge is a thing, therefore metaphysics deals with first principles of knowledge.
        \end{remark}

        \begin{definition}[Ethics]
        \label{def:ethics}
            The branch of knowledge that deals with moral principles.
        \end{definition}

        \begin{definition}[Politics]
        \label{def:politics}
            The branck of knowledge that deals with government and the state.
        \end{definition}

        \begin{definition}[Aesthetics]
        \label{def:aesthetics}
            The branch of kanowledge that deals with the principles of beauty and art.
        \end{definition}

    New concepts will be defined throughout the text.

\mainmatter

\chapter{Reality}

    \section{Existence, Identity and Consciousness as the Basic Axioms}

        The first thing we perceive when awake is very simple: something exists.
        It does not matter what you focus on, nor if your eyes are open or closed;
            if you perceive something, you perceive that something exists.
        We start from exactly this self-evidence.
        Thus, the first (and only) axiom of objectivist philosophy is Axiom~\ref{ax:existence}.
    
            \begin{axiom}[Existence]
            \label{ax:existence}
                Existence exists.
            \end{axiom}
            
        As something self-evident, existence requires no proof. Also, the concept ``existence'' cannot be defined by means of other concepts. As the widest of all concepts, referring to everything which there is, it can be defined only ostensively (by pointing at existents, for example).

        To be is to be something (specific) - to exist is to possess an identity (Definition \ref{def:identity}). This leads us to Corollary \ref{cor:identity}.
            
            \begin{definition}[Identity]
            \label{def:identity}
                \begin{enumerate}
                    \item The fact of being who or what a person or thing is.
                    \item Sum of a person or thing's attributes or characteristics.
                \end{enumerate}
            \end{definition}
        
            \begin{corollary}[Law of Identity]
            \label{cor:identity}
                To exist is to posses identity, to be is to be something. Each thing is identical with itself.
            \end{corollary}

        The fact that we perceive the existence of something, whatever it may be, makes us possess consciousness (Definition \ref{def:consciousness}). This leads us to Corollary \ref{cor:consciousness}.
        
            \begin{definition}[Consciousness]
            \label{def:consciousness}
                The faculty of being aware of that which exists.
            \end{definition}
            
            \begin{corollary}[Consciousness]
            \label{cor:consciousness}
                Consicousness exists.
            \end{corollary}
            
        Consciousness is impossible without existence: there would be nothing to be conscious of. There is never consciousness of something which is not. If something nonexistent is perceived then what the perceiver possess is not consciousness. Just as the first and only axiom (Axiom \ref{ax:existence}) the first two corollaries are also known to be true merely by sense perception, without the possibility for or need of any proof. As a matter of fact, every proof takes the axiom and corollaries above as given.
        
        Two corollaries follow from the Law of Identity: the Law of Non-Contradiction (Corollary \ref{cor:non}) and the Law of the Excluded Middle (Corollary \ref{cor:excluded}). The Law of Identity together with its two corollaries define logic (Definition \ref{def:logic}).
            
            \begin{corollary}[Law of Non-contradiction]
            \label{cor:non}
                Something never is and is not at the same time.
            \end{corollary}
            
            \begin{corollary}[Law of Excluded Middle]
            \label{cor:excluded}
                Something either is or is not, there is no third alternative.
            \end{corollary}

            \begin{definition}[Logic]
            \label{def:logic}
                Set of three laws:
                \begin{itemize}
                    \item Law of Identity,
                    \item Law of Non-Contradiction,
                    \item Law of the Excluded Middle.
                \end{itemize}
            \end{definition}
            
        Logic is an indispensable characteristic of existence. Absolutely everything that exists is bound to the laws of logic.
            
    \section{Causality as a Corollary of Identity}
        
        The content of the world mwe perceive is constituted by entities, which are everything there is to observe. The concept of ``entity'', just like the concept of ``existence'', cannot be defined by means of other concepts, just ostensively. Categories of being (such as qualities or quantities) do not have metaphysical primacy, as all represent merely aspects of entities.

        Actions (Definition \ref{def:action}) also do not exist apart from entities. There are no floating actions: there are only actions performed by entities. As we also define define ``nature'' (Definition \ref{def:nature}) we arrive to the Law of Causality (Corollary \ref{cor:causality}).
        
            \begin{definition}[Action]
            \label{def:action}
                What entities do.
            \end{definition}
            
            \begin{definition}[Nature]
            \label{def:nature}
                The attributes of an entity, what an entity is.
            \end{definition}
        
            \begin{corollary}[Law of Causality]
            \label{cor:causality}
                An entity acts in accordance with its nature.
            \end{corollary}

            \begin{remark}
                If an action is what an entity does, and its nature is its attributes, than the action, being an attribute of the entity, is in accordance with its nature. For the law of causality to not hold, an entity would need to act apart from its nature or act against it, both of which are impossible given the Law of Identity (Corollary \ref{cor:identity}). Apart from its nature, its attributes, a thing is nothing.
            \end{remark}
            
        The action of an entity is both caused (Definition \ref{def:cause}) and necessitated (Definition \ref{def:necessity}) by its nature. (Theorem \ref{the:cause_necessity}).

            \begin{definition}[Cause]
            \label{def:cause}
                \begin{enumerate}
                    \item A person or thing that gives rise to an action, phenomenon, or conditionl,
                    \item reasonable grounds for doing, thinking, or feeling something.
                \end{enumerate}
            \end{definition}
        
            \begin{definition}[Necessity]
            \label{def:necessity}
                Principle according to which something must be so.
            \end{definition}

            \begin{theorem}
            \label{the:cause_necessity}
                The action of an entity is both caused and necessitated by its nature.
            \end{theorem}

            \begin{proof}
                The nature of the entity (what it is) is what causes the entity's actions. The actions must be so, as an entity acts in accordance with its nature.
            \end{proof}
            
        Cause and effect are an universal law of reality. It is a law inherent in being qua being: to be is to be something (Corollary \ref{cor:identity}), and to be something is to act accordingly (Corollary \ref{cor:causality}). Every action has a cause, and the same cause leads to the same effect. Nevertheless, not every entity has a cause. The universe, being everything that exists, cannot have a cause, as this would require it to be caused by nonexistence - which, not being, cannot act as a cause. In other words, the universe is eternal (Theorem \ref{the:universe}).

            \begin{theorem}
            \label{the:universe}
                The universe is eternal.
            \end{theorem}
        
    \section{Existence as Possessing Primacy over Consciousness}
        
        Existence precedes consciousness. Consciousness (Definition \ref{def:consciousness}) is the faculty of being aware. An entity's nature is not affected by the nature of any other entity, even if one's nature includes the faculty of being aware of the other's existence. This fact is summarized in Corollary \ref{cor:primacy}.
        
            \begin{corollary}[Primacy of Existence]
            \label{cor:primacy}
                Things are what they are independent of consciousness.
            \end{corollary}
        
        Proofs depend on the primacy of existence, as they presuppose the principle that facts are not malleable. Since knowledge is knowledge of reality, every metaphysical principle has epistemological implications. As the nature of the world is not maleable, nothing is relevant to cognition except data drawn from the world (Chapter \ref{cha:sense_perception}). This leads us to the only valid method of cognition: reason (Chapter \ref{cha:reason})
            
    \section{The Metaphysically Given as Absolute}
    
        The discussion above culminates in the principle that no alternative to a fact of reality is possible. The metaphysically given is absolute (Corollary \ref{cor:given}).
    
            \begin{definition}[Metaphysically Given]
                Any fact inherent in existence apart from human action.
            \end{definition}
            
            \begin{definition}[Absolute]
                Necessitated by the nature of existence.
            \end{definition}

            \begin{definition}[Necessity]
                Principle according to which something must be so.
            \end{definition}
            
            \begin{corollary}
            \label{cor:given}
                The metaphysically given is absolute.
            \end{corollary}
            
        The antonym of ``necessary'' is ``chosen'', which is the case of man-made facts. Creativity (Definition \ref{def:creativity}) alters reality, but not the metaphysically given. In order for man to succeed, he needs to accept the absolute.

            \begin{definition}[Creativity]
            \label{def:creativity}
                The use of the imagination.
            \end{definition}

            \begin{definition}[Imagination]
                The faculty or action of forming new ideas, or images or concepts of external objects not present to the senses.
            \end{definition}
            
        The distinction between the given and the man-made is crucial. The given is reality, and must be accepted without evaluation. Man-made facts, on the contrary, are products of choice, and must be evaluated.  %OK
\chapter{Sense Perception and Volition}

    Metaphysics is, basically, the identification of the fact of existence and its corollaries. Epistemology, on the other hand, is the science that studies the nature and means of human knowledge. If the mind wishes to know existence, it must conform to existence. A conceptual knowledge can depart from reality, so it needs a method of cognition. Epistemology is based, therefore, on the premise that man can acquire knowledge only if he performs certain definite processes.

    Every process of knowledge involves the object of cognition and the mens of cognition. The objects is always some aspect of reality (there is nothing else to know). The means pertains to the kind of consciousness and determines the form of cognition.
    
    \section{The Senses as Necessarily Valid}
    
        We begin with a self-evidence: Corollary \ref{cor:senses}.

            \begin{corollary}[Validity of the Senses]
            \label{cor:senses}
                The senses are valid.
            \end{corollary}

            \begin{remark}
                Validity of the Senses is a corollary of Consciousnes. If man is conscious, then it is conscious of that which is; it wouldn't be conscious otherwise.
            \end{remark}
        
        Proof consists in reducing an idea back to the data provided by the senses. The data itself is, therefore, self-evident, and is outside the province of proof. Science is nothing more than the conceptual unraveling of sensory data, it has no other primary evidence from which to proceed.
        
        Our sensations are caused in part by the entities we perceive, and in part by our sense organs. Nevertheless, any difference in sensory form among perceivers is precisely that: a difference in form of perceiving the same entities, the same reality. ``Looks'' means ``appears to our visual sense''. Whatever facts the senses register are facts.
    
    \section{Sensory Qualities as Real}
        
        The entities we perceive have a nature independent of ourselves, and our sense organs also have their own nature. Nevertheless, it is possible to distinguish between form and object. The qualities of objects are not merely ``in the object'', nor are they merely ``in the mind''. The qualities we perceive are, actuallly, in man's form of grasping the object. Is is not object alone or perceiver anole, it is object-as-perceived. The form of objects is, being self-evident, a metaphisically given fact: they are real. This is Corollary \ref{cor:qualities}.

            \begin{corollary}[Reality of Sensory Qualities]
            \label{cor:qualities}
                Sensory qualities are real.
            \end{corollary}

            \begin{remark}
                Reality of Sensory Qualities is a corollary of Validy of the Senses. If our senses are valid, the form of perceiving entities our senses provide us with is also valid.
            \end{remark}

        Demanding man to perceive entities without qualities is to demand the entities to be perceived in no sensory form. It is to reject our senses because they have identity, because they exist. We can know the content of reality ``pure'', apart from man's perceptual form; but we can do so only by abstracting away man's perceptual form - only by starting from sensory data, then performing a complex scientific process.
    
    \section{Consciousness as Possessing Identity}

        Every existence is bound by the Laws of Identity and Causality (Corollaries \ref{cor:identity} and \ref{cor:causality}). This applies not only to objects, but to everything which there is, including consciousness, what leads us to Corollary \ref{cor:consciousness_identity}.
        
            \begin{corollary}
            \label{cor:consciousness_identity}
                Consciousness possesses identity.
            \end{corollary}

            \begin{remark}
                Consciouness exists (Corollary \ref{cor:consciousness}), and every existent possesses identity (Corollary \ref{cor:identity}), therefore consciousness possesses identity.
            \end{remark}
        
        Consciousness perceives what exists directly, by mens of the effects on its organs of perceptions. There is no ``more direct'' perception of reality, as this would require reality to be perceived by no means, which then would mean for it to not be perceived. The fact that human cognitive faculties have a nature is what makes them possible. Identity is the precondition of consciousness.

    \section{The Perceptual Level as the Given}

        Some animals have only sensations (as in Definition \ref{def:sensation}). We, human adults, on the other hand, encouter entities when we look at the world. This happens because we have experienced many kinds of sensations from similar objects in the past, and our brains have retained and integrated them; it has put them toguether to form an indivisible whole. This gives us the ability to see, for example, not just a brown spot, but a table.
            
            \begin{definition}[Sensation]
            \label{def:sensation}
                An irreductable state of awareness produced by the action of a stimulus on a sense organ.
            \end{definition}

        This ability exemplifies the second stage of consciousness: the perceptual level. A perception is as in Definition \ref{def:perception}.

            \begin{definition}[Perception]
            \label{def:perception}
                A group of sensations automatically retained and integrated by the brain of a living organism, which gives it the ability to be aware, not of single stimuli, but of entities.
            \end{definition}

        Direct experience means the perceptual level of consciousness. What we are given as adults when we use our senses (giving aside all conceptual knowledge) is the awareness of entities, not merely sensations. Epistemologically, therefore, the perceptual stage comes first. Perceptions constitute the base of cognition, they are the self-evident. This leads us to Corollary \ref{cor:percepts}.

            \begin{corollary}[Percepts as Given]
            \label{cor:percepts}
                The perceptual level is given.
            \end{corollary}

            \begin{remark}
                Percepts as Given is a corollary of Validy of the Senses. What our senses give us are perceptions.
            \end{remark}
    
        There is no guidance philosophy can give us about perception, it is an automatic process over which we have no power. In regard to another more complex king of integration, which we do not perform automatically, philosophy does have advice to offer. It is the integration of precepts into concepts. This brings us to the threshold of the conceptual level of consciousness and to the second issue in the anteroom of epistemology: volition.

    \section{The Primary Choice as the Choice to Focus or Not}

        Man is a volitional being (Definition \ref{def:volition}), who functions freely. A course of thought or action is free if it is selcted between alternatives, if it could be otherwise if the human had not decided as it had.

            \begin{definition}[Volition]
            \label{def:volition}
                The faculty or power of using one's will.
            \end{definition}

        For a being with volition, the primary choice is to focus or not (Theorem \ref{the:focus}).

            \begin{definition}[Primary]
            \label{def:primary}
                Irreducible.
            \end{definition}

            \begin{definition}[Focus]
            \label{def:focus}
                The state of a goal-directed mind committed to attaining full awareness of reality.
            \end{definition}

            \begin{theorem}
            \label{the:focus}
                The primary choice is the choice to focus or not.
            \end{theorem}

            \begin{proof}
                To focus is to direct the mind to a goal, is to commit oneself to attain awareness of reality, is the readiness to think, and therefore the precondition of thinking. All choices one makes by thinking can be reductible to the first choice, that of start to think about what choice to make. Therefore the irreductible choice, the primary choice, is to focus or not, as it can't be reducted to any other choice.
            \end{proof}

        To focus is work and is experienced as such. Work, in the sanse of basic menatal effort (Definition \ref{def:effort}.

            \begin{definition}[Effort]
            \label{def:effort}
                Expenditure of energy to achieve a purpose.
            \end{definition}

        A primary choice can't be explained by anything more fundamental. By its nature, it is a first cause within a consciousness.

    \section{Human Actions, Mental and Physical, as Both Caused and Free}
    
        Thought is a volition activity, its steps are chosen (as against necessitated). Aside from the primary choice to focus, the other choices are reductible. In their case, it is legitimate to ask: why did the individual choose as he did? what was the cause of his choice?

        In regard to human actions, however, to be caused does not mean to be necessitaded (as it means for matter). Man chooses the causes that shape his actions. To say that a higher-level choice was caused is to say there was a reason behind it.

        The Law of Causality (Axiom \ref{cor:causality}) still holds in the case of an irreductible choice. The action of choice is performed and necessitated by the nature of human consciousness.


    \section{Volition as Axiomatic}

        If man could not choose among alternatives, the field of epistemology would be useless, as there is no guidance philosophy can give to a deterministic process. The existence of 'validation' or 'proof' requires a volitional consciousness to chose among alternative ideas. To ask for the proof of volition is to ask to use volition to prove it (we saw an analogous situation when it is asked to prove the senses). Nevertheless, volition is self-evident, as it is something we perceive directly in ourselves, available to any act of introspection. It is a corollary of consciousness.

            \begin{corollary}[Volition]
                Human consciousness possesses volition.
            \end{corollary}

        The fact that we regularly make choices it directly accessible to us, as it is to any volitional consciousness. Volition is an axiom, a primary, a starting point of conceptual cognition and of the subject of epistemology. The faculty of reason is the faculty of volition.
  %OK 
\part{Epistemology}
\label{part:epistemology}

\chapter{Concept-Fromation}

    Sensory material is the first step of knowledge. Nevertheless, human knowledge is a conceptual phenomena. Therefore, man has to conceptualize the information provided by the senses.

    Conceptualiaztion is what enables us to go beyond the knowledge of only concretes and generalize, indentify natural laws, understand what we observe. Man won't see all trees, but can however obtain knowledge about all trees.

    A conceptual faculty determines a species' method of cognition, action and survival. To understand human knowledge one must understand concepts: what they are, how they are formed and how they are used in the quest for knowledge.

    \section{Differentiation and Integration as the Means to a Unit-Perspective}

        First, lets overview the nature of a conceptual consciousness. After man has indeitified particular entities, he can grasp relationships among these entities by grasping the similarities (Definition \ref{def:similarity}) and differences (Definition \ref{def:difference}) of their identities. These similarities are, in fact, observed in reality (Theorem \ref{the:similarities}).

            \begin{definition}[Similarity]
            \label{def:similarity}
                Partial identity.
            \end{definition}

            \begin{definition}[Difference]
            \label{def:difference}
                Partial non identity.
            \end{definition}

            \begin{theorem}
            \label{the:similarities}
                Similarities and differences are perceptually given.
            \end{theorem}

            \begin{proof}
                Just like the identity of an existence is observed in reality, is perceptually given, so are partial identities and partial non identities between existents.
            \end{proof}

        When we look at something, we do not se a thing, we see a thing of a specific kind, in relatioship to every other thing of the same kind (for example, you don't see this thing, you see this book, this object, this document). We grasp an entity as a member of a group of similar members. The implicit concept present in this view of the world is "unit", as in Denifition \ref{def:unit}. The ability to regard entities as units is man's distinctive method of cognition.

            \begin{definition}[Unit]
            \label{def:unit}
                An existent regarded as a separate member of a group of two or more similar members.
            \end{definition}

        By treating entities as members of groups of similar entities, man can apply to all entities of a group the knowledge he gains by studying only some of its members. It is essential to grasp that in the world apart form men there are no units, there are only existents. To view things as units is to adopt a human perspective on things.

        The concept "unit" is an act of consciousness, but it is not an arbitraty creatin of consciousness: it is a method of identification or classification according to the attributes which a consciousness observes in reality. Units do no exsit qua units, what exists are things, but units are things viewed by a consciousness in certain existing relationships.

        Without the implicit concept of "unit" man could not reach the conceptual method of knowledge. Without the same implicit conpect man could not enter the field of mathematics. Thus the same concept is the base and start of two fields: the conceptual and the mathematical, a fact that points to an essential connection between the two fields. It suggests that concept-formation is in some way a mathematical process.

        Two main processes are involved in the concious process man perform in order to regard entities as units: differentiation and integration (Definitions \ref{def:differentiation} and \ref{def:integration}).

            \begin{definition}[Differentiation]
            \label{def:differentiation}
                The pocess of grasping differences.
            \end{definition}

            \begin{definition}[Integration]
            \label{def:integration}
                The process of uniting elements into an inseparable whole.
            \end{definition}

        In order to move from the stage of sensation to that of perception, we have to discriminate sensory qualities and integrate them into entities. The same two processes occur in the movement from percepts to concepts. In this case, however, the processes are not performed for us automatically.

        We begin the formation of concepts by isolating a group of concretes. We do this on the basis of observed similarities that distibguish this concretes from the rest of our perceptual field. It is important to stress that the similarities are observed, they are perceptually given (Theorem \ref{the:similarities}).

        \begin{proof}
            Following Definition \ref{def:similarity}, similarity is partial identity. Identity is self-evident, therefore perceptually given.
        \end{proof}

        The distinctively human element in the above is our ability to abstract (Definition \ref{def:abstraction}) such similarities from the differences in which they are embedded.

            \begin{definition}[Abstraction]
            \label{def:abstraction}
                The power of selective focus and treatment; it is the power to separate mentally and make cognitive use of an aspect of reality that cannot exist separately.
            \end{definition}

        Man makes something out of the similarities he observes: he makes such data the basis of a method of cognitive organization. The first step of the method is the mental isolation of a group of similars.

        But an isolated perceptual group is not yet a concept. To achieve a cognitive result, we must proceed to integrate. When we integrate entities into an iseparable whole such a whole is a new entitiy, a mental entity. This entity stands for an unlimited number of concretes, including the ones we have not yet observed.

        The tool that makes this kind of integration possible is language. A word is the only form in which man's mind is able to retain such a sum of concretes. A word (Definition \ref{def:word}) is a concrete, perceptually graspable symbol.

            \begin{definition}[Word]
            \label{def:word}
                A symbol that denotes a concept.
            \end{definition}

        Only concretes exist. If a concept is to exist, therefore, it must exist in some way as a concrete. It is not true that words are necessary primarily for the sake of communication. Words are essential to the process of conceptualization and thus to all thought. A word without a concept is noise. Words transform concepts into (mental) entities, definitions provide them with identity.

        Now let us consider a important problem: the relationship of concepts to existents. A percept is a direct awareness of an existing entity, but a concept involves a process of abstraction. A concept refers to what all the concretes in a given class possess in commom. The problem is: what is this attribute and how does one discover it?

        All along we have been using concepts to reach the truth. Now we must turn to the precondition of this use and face the fundamental problem of epistemology. We must ground concepts themselves in the nature of reality.

    \section{Concept-Formation as a Mathematical Process}

            \begin{definition}[Mathematics]
            \label{def:mathematics}
                Science of measurement.
            \end{definition}

            \begin{definition}[Measurement]
            \label{def:measurement}
                The identification of a relationship — a quantitative relationship established by means of a standard that serves as a unit.
            \end{definition}

        Measurement involes two concretes: the existent being measured and the existent that is the standard of measurement. In the process of measurement, we identify the relationship of any instance of a certain attribute to a specific instance of it selected as the unit. The former may range across the entire spectrum of magnitude; the latter, the (primary) unit, must be within the range of human perception.

        Similar concretes integrated by a concept differ from one another only quantitatively, only in the measurements of their characteristics. When we form a concept, therefore, our mental process consists in retaining the characteristics, but omitting their measurements. 
        
        The principle is: the relevant measurements must exist in some quantity, but may exist in any quantity. In this sense, in the form of an epistemological standing order, the concept may be said to retain all the characteristics of its referents and to omit all the measurements. A man's grasp of similarity is actually his mind's grasp of a mathematical fact: the fact that certain concretes are commensurable — that they (or their attributes) are reducible to the same unit(s) of measurement.

        So far, we have been considering measurement primarily in regard to the integration of concretes. Measurement also plays a special role in the first step of concept-formation: the differentiation of a group from other things. Such differentiation cannot be performed arbitrarily, only by a commensurable characteristic. 
        
        To differentiate existents, we define the concept of Concetual Commom Denominator (Definition \ref{def:ccd}). We also define concept (Definition \ref{def:concept}).

            \begin{definition}[Conceptual Common Denominator]
            \label{def:ccd}
                The characteristic(s) reducible to a unit of measurement, by means of which man differentiates two or more existents from other existents possessing it.
            \end{definition}

            \begin{definition}[Concept]
            \label{def:concept}
                A mental integration of two or more units possessing the same distinguishing characteristic(s), with their particular measurements omitted.
            \end{definition}

        A concept is not a product of arbitraty choice, it has a basis on and do refer to the facts of reality. A concept denotes facts - as processed by a human consciousness.

            \begin{theorem}
                A concept has a basis on and do refer to facts of reality.
            \end{theorem}

            \begin{proof}
                Perception makes us aware of the existents (facts of reality). Once we observe similarities (which are also facts of reality) between existents, we are able to differentiate them from all other existents, and finally integrate them into an indivisible whole.
            \end{proof}

        What the window of mathematics reveals is not the mechanics of deduction, but of induction. 

    \section{Definition as the Final Step in Concept-Formation}

        The final step in concept-formation is definition.

        The perceptual level of consciousness is automatically related to reality. A concept, however, is an integration that rests on a process of abstraction. Such a mental state is not automatically related to concretes, as is evident from the many obvious cases of ``floating abstractions'' (Definition \ref{def:floating}).

            \begin{definition}[Floating Abstraction]
            \label{def:floating}
                A concept detachets from existents.
            \end{definition}

        If a concept is to be a device of cognition, it must be tied to reality. It must denote units that one has methodically isolated from all others. This is the basic function of a definition: to distinguish a concept from all other concepts and thus to keep its units differentiated from all other existents.

        A definition cannot list all the characteristics of the units. Instead, a definition identifies a concept's units by specifying their essential characteristics (Definition \ref{def:essential_characteristic}) 

            \begin{definition}[Essential Characteristic(s)]
            \label{def:essential_characteristic}
                The fundamental characteristic(s) which makes the units the kind of existents they are and differentiates them from all other known existents.
            \end{definition}

        The distinguishing characteristic of an entity is the \textit{differentia}, the concetres from which we are distinguishing the entity from give rise to the \textit{genus}.

        Another such feature is the fact that definitions, like concepts, are contextual (Definition \ref{def:context}). The concepts' definitions may change as the context changes.

            \begin{definition}[Context]
            \label{def:context}
                \begin{enumerate}
                    \item The entire field of a mind's awareness or knowledge at any level of its cognitive development.
                    \item The sum of cognitive elements conditioning an item of knowledge.
                \end{enumerate}
            \end{definition}

        When a definition is contextually revised, the new definition does not contradict the old one. The knowledge earlier gained remains knowledge. What changes is that, as one's field of knowledge expands, these facts no longer serve to differentiate the units.

        Definitions (like all truths) are  empirical statements. They derive from certain kinds of observations — those that serve a specific (differentiating) function within the conceptualizing process.

        The definition must state the feature that most significantly distinguishes the units; it must state the fundamental. 

            \begin{definition}[Fundamental]
            \label{def:fundamental}
                The characteristic responsible for all the rest of the units' distinctive characteristics, or at least for a greater number of these than any other characteristic is.
            \end{definition}

        The definitional principle is: wherever possible, an essential characteristic must be a fundamental.

        The truth of a proposition depends not only on its relation to the facts of the case, but also on the truth of the defi- nitions of its constituent concepts. If these concepts are detached from reality then so are the propositions that employ them. A proposition can have no greater validity than do the concepts that make it up. The precondition of the quest for truth, therefore, is the formulation of proper definitions.

        A concept is not interchangeable with its definition. A concept designates existents, including all their characteristics, whether definitional or not.

        It is crucially important to grasp the fact that a concept is an ``open-end'' classification which includes the yet-to-be-discovered characteristics of a given group of existents. One important implication of the above is that a concept, once formed, does not change. The knowledge men have of the units may grow and the definition may change accordingly, but the concept, the mental integration, remains the same. 






  %OK
\chapter{Objectivity}

    We will now begin to identify the rules men must follow in their thinking if knowledge is the goal. These rules can be condensed into one principle: thinking, to be valid, must adhere to reality. But how does one guarantee adherence to reality? The answer lies in the concept of ``objectivity''.

    \section{Concepts as Objective}

        ``Objectivity'' arises because concepts are formed by a specific process and, as a result, bear a specific kind of relationship to reality. Concepts do not pertrain to consciousness alone or to existence alone, they are products of a specific kind of relationship between the two. Abstractions are products of man's faculty of cognition, which, concerned with grasping reality, must adhere to it.

        Conceptualization is not an automatic reaction to stimuli (as is perception). Concept formation is volitional, requiring effort. Man must learn to do it correctly. In such processing, the basic method he uses, measurement-omission, is dictated by the nature of his cognitive faculty. The result is a human perspective on things, not a revelation of a special sort of entity or attribute intrinsic in the world apart from man.

        On the other hand, consciousness is the faculty of grasping that which is, and there is a metaphysical basis for concepts. The charateristics of entities is a fact, not a creation of man. The method of concept formation conforms each step to facts, otherwise it would be irrelevant to a cognitive need.

        We define two classes of concepts wich are not objective: stolen concepts, which is the use of a higher-level concept while denying or ignoring its hierarchical roots; invalid concepts, which are words without specific definitions, without referents (the test of an invalid concept is the fact that it cannot be reduced to the perceptual level).

    \section{Objectivity as Volitional Adherence to Reality by a Method of Logic}

        The objective approach to concepts leads to the view that, beyond the perceptual level, knowledge is the grasp of an object through an active, reality-based process chosen by the consciousness. The steps of this process must contitute a method of cognition, that guarantees that men remains in contact with reality. We define objectivity in Definition \ref{def:objectivity}. Reality, existents, cannot be objetive, they simply are. It is conceptual processes which are objective.

            \begin{definition}[Obectivity]
            \label{def:objectivity}
                Volitional adherence to reality by following certain rules of method, a method based on facts and appropriate to man's form of cognition.
            \end{definition}

        The method of cognition that objectivity requires is logic (Definition \ref{def:logic}), which is a volitional consciousness' method of conforming to reality - it is the method of reason. %Logic is the art of noncontradictory identification.

    \section{Knowledge as Contextual}

        Concepts are a relational form of knowledge. In other words, concepts are formed in a context — by relating concretes to a field of contrasting entities. This body of relationships, which constitutes the context of the concept, is what determines its meaning. Human knowledge on every level is relational. Knowledge is not a juxtaposition of independent items; it is a unity.

        %The relational nature of knowledge derives from two roots, one pertaining to the nature of existence, the other, to the nature of consciousness.
        
        Metaphysically, there is only one universe. This means that everything in reality is interconnected. Knowledge, therefore, which seeks to grasp reality, must also be a total; its elements must be interconnected to form a unified whole reflecting the whole which is the universe.

        Leaving aside the primaries of cognition, which are selfevident, all knowledge depends on a certain relationship: it is based on a context of earlier information.

        %\begin{theorem}[Knowledge is contextual]
        %    \label{the:context}
        %        Metaphysically, there is only one universe. Therefore everything is interconnected. Knowledge, as it seeks to grasp reality, is also interconnected. The truth of any statement depends on the truth of every other, which is the context of the first.
        %\end{theorem}

    \section{Knowledge as Hierarchical}

        A first-level concept is one formed directly from perceptual data. Higher-level concepts, by contrast, presuppose earlier concepts. A definite order of concept-formation in necessary. We begin with those abstractions that are closest to the perceptually given and mode gradually away from them.

        The same principle of order applies to ecery field of human knowledge, nor merely to concept-formation - knowledge has a hierarchical structure. A hierarchy of knowledge means a body of concepts and conclusions ranked in order of lofical dependence.

            %\begin{definition}[Hierarchy]
            %    A body of persons or things ranked in grades, orders, or classes, one above another.
            %\end{definition}

        The cencept or hierarchy is epistemological, not metaphysical, as facts are simultaneous in reality.

        An order of logical dependence exists from man's perspective, because man cannot come to know all facts with the same directness. In some but not all cases, the hierarchy of human knowledge depends on the nature of man's senses -  on the type of information they provide.

        The hierarchical view of knowledge states not only that every (nonaxiomatic) item has a context, but also that such context itself has an inner structure of logical dependence.

        The epistemological responsibility imposed on man by the fact that knowledge is contextual is the need of integration. The responsibility imposed by the fact that knowledge is hierarchical is the need of reduction. Proof is a form of reduction, of retracing the hierarchical steps of the learning process.
  %OK
\chapter{Reason}

    The whole of our phisolophy amounts to ``follow reason''. ``Reason'' (Definition \ref{def:reason}), nevertheless, is a higher-level concept, and to grasp its meaning one must first grasp its hierarchical roots. These are what we have been discussing about in the chapters above.

        \begin{definition}[Reason]
        \label{def:reason}
            Method of cognition that proceeds in accordance with facts, which are established, directly or indirectly, by observation.
            
            The faculty that:
            \begin{itemize}
                \item identifies and integrates the material provided by man's senses;
                \item enables man to discover the nature of existents — by virtue of its power to condense sensory information in accordance with the requirements of an objective mode of cognition;
                \item organizes perceptual units in conceptual terms by following the principles of logic.
            \end{itemize}
        \end{definition}

        \begin{remark}
            The latter formulation highlights the three elements essential to the faculty: its data, percepts; its form, concepts; its method, logic.
        \end{remark}

    Reason is the existence-oriented faculty. Accepting reason is accepting reality.

    \section{Emotions as a Product of Ideas}
    
        First, let's distinguish emotion from sensation. Sensation follows Definition \ref{def:sensation}. It is automatic, independent of ideas. An emotion (Definition \ref{def:emotion}), on the other hand, is dependent on ideas.

            \begin{definition}[Emotion]
            \label{def:emotion}
                A state of consciousness which is a response to an object one perceives (or imagines).
            \end{definition}

        An emotion arises only if two necessary conditions are met:

        \begin{enumerate}
            \item one must identify the object perceived (or imagined),
            \item one must evaluete the object.
        \end{enumerate}
    
        An individual cannot have an emotional response to and object without identity, that means nothing. A nothing cannot be evaluated. He also cannot have an emotion driven by something de does not evaluate. Seomthing not evaluated may have identity, but does not have any meaning.
    
        There are four steps in the generation of an emotion:

        \begin{enumerate}
            \item perception (or imagination),
            \item identification,
            \item evaluation,
            \item reponse.
        \end{enumerate}
    
        Only the first and last step are necessarily conscious. The other two may occur without the need of conscious awareness, as once an individual has formed value-judgments, he automatizes them.
        
        Value-judgments are formed by thoughts, and ultimately on a philosophic view of oneself and others; of man, life and the universe. Such a view conditions all one's emotions. These views may be held implicitly of implicitly.
        
        What makes emotions incomprehensible to many people is the fact that their views are not only implicit, but contracditory. This leads to the appearance of a conflict between thought and feelings.

    \section{Reason as Man's Only Means of Knowledge}
    
            \begin{theorem}
                Reason is man's only means of knowledge
            \end{theorem}

            \begin{proof}
                There are two candidates for means of knowledge: reason and emotion.

                Reason, following Definition \ref{def:reason} is a method of cognition that proceeds in accordance with facts, which are established, directly or indirectly, by observation. Emotion, on the other hand, following Definition \ref{def:emotion}, is a state of consciousness which is a response to an object one perceives (or imagines).

                Emotion is the result of past conclusions. It, consequently, cannot be trust as a means to knowledge, as our past conclusios may be wrong. Reason is exactly what identifies the true and false whithin our thougths, as it compares them to the facts.
            \end{proof}
            
        Reason is a faculty of awareness; its function is to perceive that which exists by organizing observational data. And reason is a volitional faculty; it has the power to direct its own actions and check its conclusions, the power to maintain a certain relationship to the facts of reality. Emotion, by contrast, is a faculty not of perception, but of reaction to one's perceptions. This kind of faculty has no power of observation and no volition; it has no means of independent access to reality, no means to guide its own course, and no capacity to monitor its own relationship to facts.
        
        Emotions are automatic consequences of a mind's past conclusions. Feeling follows obediently. It has no power to question its course or to check its roots against reality. Only man's volitional, existence-oriented faculty has such power.
        
        Now, through a study of man's means of consciousness, this earlier discussion has been confirmed and completed. Metaphysics and epistemology unite. They unite in declaring that "emotions are not tools of cognition.
        
        The conclusion is clear: there is no alternative or supplement to reason as a means of knowledge. If one attempts to give emotions such a role, then he has ceased to engage in the activity of cognition.
        
        If an individual experiences a clash between feeling and thought, he should not ignore his feelings. He should identify the ideas at their base (which may be a time-consuming process); then compare these ideas to his conscious conclusions, weighing the conflicts objectively; then amend his viewpoint accordingly, disavowing the ideas he judges to be false
        
        The above indicates the pattern of the proper relationship between reason and emotion in a man's life: reason first, emotion as a consequence.

    \section{The Arbitrary as Neither True or False}
    
        An arbitrary claim is one for which there is no evidence, either perceptual or conceptual. an arbitrary claim is automatically invalidated. The rational response to such a claim is to dismiss it, without discussion, consideration, or argument
        
        An arbitrary statement has no relation to man's means of knowledge. Since the statement is detached from the realm of evidence, no process of logic can assess it
        
        Since it is affirmed in a void, cut off from any context, no integration to the rest of man's knowledge is applicable; previous knowledge is irrelevant to it. Since it has no place in a hierarchy, no reduction is possible, and thus no observations are relevant.
        
        If an idea is cut loose from any means of cognition, there is no way of bringing it into relationship with reality
        
        An arbitrary idea must be given the exact treatment its nature demands. One must treat it as though nothing had been said. The reason is that, cognitively speaking, nothing has been said. One cannot allow into the realm of cognition something that repudiates every rule of that realm.
        
        The true is identified by reference to a body of evidence; it is pronounced "true" because it can be integrated without contradiction into a total context. The false is identified by the same means; it is pronounced "false" because it contradicts the evidence and/or some aspect of the wider context. The arbitrary, however, has no relation to evidence or context; neither term, therefore—"true" or "false"—can be applied to it.
        
        The onus of proof rule states the following. If a person asserts that a certain entity exists (such as God, gremlins, a disembodied soul), he is required to adduce evidence supporting his claim. If he does so, one must either accept his conclusion, or disqualify his evidence by showing that he has misinterpreted certain data. But if he offers no supporting evidence, one must dismiss his claim without argumentation, because in this situation argument would be futile. It is impossible to "prove a negative," meaning by the term: prove the nonexistence of an entity for which there is no evidence
        
        a nonexistent is nothing; it is not a constituent of reality, and it has no effects. If gremlins, for instance, do not exist, then they are nothing and have no consequences. In such a case, to say: "Prove that there are no gremlins/' is to say: "Point out the facts of reality that follow from the nonexistence of gremlins." But there are no such facts. Nothing follows from nothing.

    \section{Certainty as Contextual}
    
        Human knowledge is limited. Logical processing of an idea within a specific context of knowledge is necessary and sufficient to establish the idea's truth.
        
        Consciousness has identity, and epistemology is based on the recognition of this fact. Epistemology investigates the question: what rules must be followed by a human consciousness if it is to perceive reality correctly? Nothing inherent in human consciousness, therefore, can be used to undermine it

        If a fact is inherent in human consciousness, then that fact is not an obstacle to cognition, but a precondition of it—and one which implies a corresponding epistemological obligation

        Man is a being of limited knowledge—and he must, there fore, identify the cognitive context of his conclusions.

        The im plicit or explicit preamble to his conclusion must be: "On the basis of the available evidence, i.e., within the context of the factors so far discovered, the following is the proper conclu sion to draw.

        If a man follows this policy, he will find that his knowl edge at one stage is not contradicted by later discoveries.

        The advanced conclusions augment and enhance his earlier knowledge; they do not clash with or annul it.

        The principle here is evident: since a later discovery rests hierarchically on earlier knowledge, it cannot contradict its own base.
        
        Although the researchers cannot claim their discovery as an out-of-context absolute, they must treat it as a contextual absolute (i.e., as an immutable truth within the specified con text).

        A man does not know everything, but he does know what he knows.
        
        This is why there can be no such thing as "some evi dence" in favor of an entity transcending nature and logic The term "evidence" in this context would be a stolen con cept. Since nothing can ever qualify as a "proof" of such an entity, there is no way to identify any data as being a "part proof" of it, either.  %OK
\part{Ethics}
\label{part:ethics}

\chapter{Man}

    There is no question more crucial to man than the question: what is man? What kind of being is he? What are his essential attributes?

    A philosophical inquiry into man is not part of the special sciences: it is a study of man's metaphysical nature.

    In this inquiry, one is not concerned to discover what is right for man or wrong, desirable or undesirable, good or evil. The concern here is a purely factual question: what is the essence of human nature?

    Ultil you know what you are you cannot know what you ought to do. The root of all evaluative sciences is the nature of man.

    \section{Man as Conditional and Goal-Directed}

        Man is a living being, which means that he can die. His existence is not given: it is conditional (Definition \ref{def:conditional}).

            \begin{definition}[Conditional]
            \label{def:conditional}
                Subject to one or more conditions or requirements being met; made or granted on certain terms.
            \end{definition}

        Man's existence is subject to requirements. It requires a specific course of action, which itself requires effort. The actions of man (his efforts) are goal-directed (Definition \ref{def:goal}) if he is to live.

            \begin{definition}[Goal]
            \label{def:goal}
                The object of a person's ambition or effort; an aim or desired result.
            \end{definition}

    \section{Reason as Man's Basic Means of Survival}

            \begin{theorem}
                Reason is man's basic means of survival.
            \end{theorem}

            \begin{proof}
                Man, if he is to survive, needs to produce what his survival requires. In order to produce objects in reality he needs to deal with reality. For man's actions to be successful in dealing with reality, he needs to know reality. The method of knowing reality is reason. Reason, therefore, is man's basic tool of survival.
            \end{proof}


    \section{Reason as an Attribute of the Individual}

        Reason is an attribute of the individual, There is no such thing as a collective mind or brain. Thought is a process that must be initiated and directed at each step by the choice of one man, the thinker. Only an individual qua individual can perceive, abstract, define, connect.  %OK
\chapter{The Good}

    Ethis is the branch of philisophy that deals with moral principles (Defintion \ref{def:ethics}). Moral principles are to guide man's choices and actions - which determine the purpose and course of his life. Man needs a moral code because his life requires a specific course of action and, being a conceptual entity, he cannot follow this course except by the guidance of concepts.

    There are two key questions aswered by ethics:
    \begin{itemize}
        \item Fow what a man should live?
        \item By what fundamental principle should he act in order to achieve this end?
    \end{itemize}

    These quetions determine the ultimate value and the primary virtue. Our answers, at the end of this chapter, will be the following: the ultimate value is life (oneself's), and the primary virtue is rationality.

    The main problem of ethics, as a brach of phiolosophy that follows metaphysics and epistemology is: how can knowledge about what is bring to the knowledge of what ought to be? We will hold in this chapter that facts do lead to values.
    
    Ehics can be validated objectively, it is a science (it can be discovered with cognition) - and a human necessity.

    \section{``Life'' as the Essential Root of ``Value''}

        First, lets define ``value'' (Definition \ref{def:value}) and consequently define ``important'' (Definition \ref{def:important}).

            \begin{definition}[Value]
            \label{def:value}
                A person's principles or standards of behavior; one's judgment of what is important in life.   
            \end{definition}

            \begin{definition}[Important]
            \label{def:important}
                Of great significance or value; likely to have a profound effect on success, survival, or well-being.
            \end{definition}

        Now we are able to prove that life is the ultimate value (Theorem \ref{the:ultimate}), and, accordingly, that remaining alive is the goal of values and of all proper action.

            \begin{theorem}
            \label{the:ultimate}
                ``Life'' is the ultimate value.
            \end{theorem}

            \begin{proof}
                A value is something important - something that helps survival (the maintence of life). Survival - life - is the ultimate value. The most important thing in life is life itself.
            \end{proof}

        Value pressuposes goal-directed action (behaviour). An object is outside the field of value if action in relation to it is inapplicable or ineffectual. Living organisms, therefore, are the entities that make value possible. They, nevertheless, do not exist in order to pursue values: they pursue values in order to exist.

        Only self-preservation can be an ultimate goal, which serves no end beyond itself. Philosophically speaking, the essence of self-preservation is: accepting the realm of reality.

    \section{Man's Life as the Standard of Moral Value}

       Man has a nature, he must follow a specific course of action if he is to survive. But man is not born knowing what that course is, nor does such knowledge well up in him effortlessly. To know this course is the purpose of morality (Definition \ref{def:morality}).

            \begin{definition}[Morality]
            \label{def:morality}
                \begin{enumerate}
                    \item Principles concerning the distinction between right and wrong or good and bad behavior,
                    \item a particular system of values and principles of conduct, especially one held by a specified person or society.
                \end{enumerate}
            \end{definition}

            \begin{definition}[Principle]
                \label{def:principle}
                    A fundamental truth or proposition that serves as the foundation for a system of belief or behavior or for a chain of reasoning.
            \end{definition}

        Man (and only man) needs morality in order to survive. Moral laws are principles that define how to nourish and sustain human life. It is the science of human self-preservation. Man needs a moral code because his life requires a specific course of action and, being a conceptual entity, he cannot follow this course except by the guidance of concepts.
        
        No action an organism takes is irrelevant to its existence, therefore, morality applies to every action. For a rational being, principled action is the only effective kind of action.

        A code of values based on reason and proper to man must hold life as the ultimate value. All that which is proper to the life of a rational being is the good; all that which destroys it is the evil.

    \section{Rationality as the Primary Virtue}

        What are the principles of human survival? What objects must man hold as values if he is to preserve his life, and what virtues must he practice in order to achieve them?

        The faculty of reason is man's basic tool of survival. The primary choice is to exercise this faculty or not. If life is the standard, therefore, the basic moral principle is obvious. It tells us the proper evaluation of reason.

        To live, man must hold three things as the supreme and ruling values of his life: Reason—Purpose—Self-esteem. Reason, as his only tool of knowledge—Purpose, as his choice of the happiness which that tool must proceed to achieve—Self-esteem, as his inviolate certainty that his mind is competent to think and his person is worthy of happiness, which means: is worthy of living. These three values imply and require all of man's virtues...

        Epistemology tells us reason is valid. Ethics tells us reason is a value - is important. If one choses to live, one must hold reason as a value.

            \begin{definition}[Virtue]
            \label{def:virtue}
                The action by which one gains and keeps a value.
            \end{definition}

        "Rationality," according to Ayn Rand, is "the recognition and acceptance of reason as one's only source of knowledge, one's only judge of values and one's only guide to action."

        This means the application of reason to every aspect of one's life and concerns. It means choosing and validating one's opinions, one's decisions, one's work, one's love, in accor- dance with the normal requirements of a cognitive process, the requirements of logic, objectivity, integration. Put nega- tively, the virtue means never placing any consideration above one's perception of reality. 

        By the same token, there is only one primary vice, which is the root of all other human evils: irrationality. This is the deliberate suspension of consciousness, the refusal to see, to think, to know.

        Now let me con- sider certain aspects of rationality in greater detail.

        To begin with, one cannot follow reason unless one ex- ercises it. Rationality demands continual mental activity. Rationality requires the systematic use of one's intelligence.

        The point is not that one must become a genius or even an intellectual. Contrary to a widespread fallacy, reason is a faculty of human beings, not of "supermen." The moral point here is always to grow mentally, to increase one's knowledge and expand the power of one's consciousness to the extent one can, whatever one's profession or the degree of one's intelligence.

        The men of virtue are the men who choose to practice and welcome this kind of struggle on principle, as a lifelong commitment.

        Just as every idea has a relationship to one's other ideas, and none can be accepted until it is seen to be an element of a single cognitive whole; so every fact has a relationship to other facts, and none can be evaded without tearing apart and destroying that kind of whole.

        The existential side of rationality is the policy of acting in accordance with one's rational conclusions. There is no point in using one's mind if the knowledge one gains thereby is not one's guide in action.

        This aspect of rationality subsumes several obligations.29 It requires that one choose not only his abstract values but also his specific goals by a process of rational thought. It requires that one know what his motives are. It requires that one choose the means to his ends by reference to explicitly defined principles, both moral and scientific. And it requires that one then enact the means, accepting the law of causality in full.

        In epistemology, we concluded that emotions are not tools of cognition. The corollary in ethics is that they are not guides to action.

        The proper approach in this issue is not reason versus emotion, but reason first and then emotion. This approach, as we have seen, leads to the harmony of reason and emotion, which is the normal state of a rational man. 

    \section{The Individual as the Propor Beneficiary of his Own Moral Action}

        Each individual must choose his values and actions by the standard of man's life—in order to achieve the purpose of maintaining and enjoying his own life.

        Thus Objectivism advocatesegoism—thepursuit of self-interest—the policy of selfishness.

        It simply states: whatever man's proper self-interest consists of, that is what each individual should seek to achieve.

        In the Objectivist view, the validation of egoism consists in showing that it is a corollary of man's life as the moral standard.

        "Only the alternative of life vs. death," I said earlier, "creates the context for value-oriented action. . . ." And "only self-preservation," I said, "can be an ultimate goal.

        "Egoistic," in the Objectivist view, means self-sustaining by an act of choice and as a matter of principle.

        The wider principle demanding such egoism is the fact that survival requires an all-encompassing course of action. If an action of his is not for his life, then, as we have seen, it is against his life

        [...]
        
    \section{Values as Objective}

        Values, like concepts, are not intrinsic or subjective, but objective. Values are not intrinsic features of reality, they require a valuer - the good is not good in itself, it is good to man and for the sake of he's life. Values are neither subjective, they require the observation of reality - the realm of facts is what creates the need to choose a certain goal.

        Moral value does not pertain to reality alone or to consciousness alone, it arises because a certain kink of entity - a living, volitional, conceptual organism - sustains a certain realtionship to the external world. The good is an aspect of reality in relationship to man. The good designates facts - the requirements of survival - as identified conceptually.

        Morality is  a must - if; it is the price of the choice to live. That choice itself, is not a moral choice, it preceded morality; it is the decision of consciousness that uderlies the need of morality.

        Suicide, nevertheless, is sometimes justified.




          %OK
\chapter{Virtue}

    Above we argued rationality is the primary virtue. Now, we turn to the derivative virtues of rationality.

    \section{Independence as a Primary Orientation to Reality, not to Other Men}

        \begin{definition}[Independence]
        \label{def:independence}
            One's accpetance of the responsibility of forming one's own judgments and of living by the work of one's own mind.
        \end{definition}

        Rationality is the primary virtue, and reason is a faculty of the individual. Therefore, in order to be rational, one must think independently (a pleonasm). This means not accepting thoughts or jugdments of others, but to form one's own. This does not mean never thinking like other people or accepting ideas created by other, but it does mean that this acceptance must be rational, the thoughts must be integrated by an act of reason - thinking.

    \section{Integrity as Loyalty to Rational Principles}

        \begin{definition}[Integrity]
        \label{def:integrity}
            Loyalty in action to one's convictions and values.
        \end{definition}

        Independence is forming one's own ideas. Integrety is following one's own ideas, acting rationally. Reality, to be mastered, must be known, but it also requires that man acts on his knowledge. There is no use for knowledge if it is not to be acted upon.

    \section{Honesty as the Rejection of Unreality}

        

    \section{Justice as Rationality in the Evaluation of Men}

        

    \section{Productiveness as the Adjustment of Nature to Man}

        \begin{definition}[Productiveness]
        \label{def:productiveness}
            The process of creating material values.
        \end{definition}

        Productiveness is a necessity of human survival. Just as there is not such a thing as ``too much'' knowledge, there is not such a thing as ``too much'' wealth. Intelectually, every discovery enhances human life; existentially, every material achievement also enhances human life.

        Productiveness is the creation of \textit{material} values, not because knowledge is important, but because it is important exclactly because it is useful in creating material values.

    \section{Pride as Moral Ambitiousness}

        \begin{definition}[Pride]
            A feeling of deep pleasure or satisfaction derived from one's own achievements, the achievements of those with whom one is closely associated, or from qualities or possessions that are widely admired.
        \end{definition}

        \begin{definition}[Self-esteem]
            Confidence in one's own worth or abilities; self-respect.
        \end{definition}

        Pride and self-esteem are results of virtue, of moral action.


    \section{The Initiation of Physical Force as Evil}

        Initiation of physical force is the destruction of rationality, and therefore of every other virtue and of every value. To force a man to do something is to deny him the possibility of acting on reason, thus turning any virtuous action (value achievement) impossible.
        
        A (rational) victim of force still thinks what he thinks, as force cannot change the content of one's mind. Thus the victim of force acts \textit{against} his judgement, and his cognition becomes useless. As a rational mind acts virtuously, to act against its judgement is to act perversely, to act against one's own life. Initiating force, when it does not kill the victim, kills his capacity to live.

        Coercion places the individual in an impossible metaphysical condition. If he does not act in accordance with his conlusions, he is doomed by reality. If he does, he is doomed by the forcer.

        Finally, one important thing to notice is that only initiation - not retaliation - of physical force is evil. While initiation of physical force destroys rationality, retaliation protects it. It cannot creates values, however - it can only protect them.  %---3,4
\chapter{Happiness}

    The existential reward of virtue is life, and the emotional reward is happiness. Happiness is the \textit{only} moral purpuse of man's life.

    \section{Virtue as Practical}

        \begin{definition}[Practical]
        \label{def:practical}
            \begin{enumerate}
                \item Of or concerned with the actual doing or use of something rather than with theory and ideas.
                \item (Of an idea, plan, or method) likely to succeed or be effective in real circumstances; feasible.
            \end{enumerate}
        \end{definition}

        The moral man's concept of the good - life - is his fundamental standard of practicality.

    \section{Happiness as the Normal Condition of Man}

        There are two oposite emotions - the first follows the achievement of value, the other from its loss - they are joy and suffering (Definitions \ref{def:joy} and \ref{def:suffering}). Happiness (Definition \ref{def:happiness}) follows from joy.

        \begin{definition}[Joy]
        \label{def:joy}
            Emotion that follows from the achienvement of values.
        \end{definition}

        \begin{definition}[Suffering]
        \label{def:suffering}
            Emotion that follows from the loss of values.
        \end{definition}

        \begin{definition}[Happiness]
        \label{def:happiness}
            A state of noncontradictory joy.
        \end{definition}

        To feel happiness, it is necessary to achieve noncontradictory - rational - values. Therefore the moral, the practical and the happy are the same. To feel happiness is to feel a metahphysical pleasure: the pleasure of being in the right relationship to existence. Happiness is the purpose of ethics, it is an end in itself. A rational man must fight for and expect happiness.

        While joy and suffering follow from virtue, which pressuposes evaluation, from evaluation itself love and fear (Definitions \ref{def:love} and \ref{def:fear}) follow.

        \begin{definition}[Love]
        \label{def:love}
            Emotion that follows from the appreciation of values.
        \end{definition}

        \begin{definition}[Fear]
        \label{def:fear}
            Emotion that follows from the depreciation of disvalues.
        \end{definition}

    \section{Sex as Metaphysical}

        Here we shall consider sex as in the life of a rational man.

        To celebrate something is to focus on it. Sex is a celebration of love. Love itself is the appreciation of values. Since the value of a man is his virtue and the latter is his volition, which is who he truly is, to celebrate self love is to celebrate oneself. To celebrate love towards the other is also to celebrate who the other truly is.

          %OK
\part{Politics}
\label{part:politics}

\chapter{Government}

    Politics, like ethics, is a normative branch of philosophy. Politics defines the principles of a proper social system, including the proper functions of government. Politics rests on ethics (and thus on metaphysics and spistemoloy); it is an application of ethics to social questions.

    What type of society conforms to or reflects the principles of morality? - this is the question asked by philosophicas politics. In Objectivism, the question is: what type of society conforms to the requirements of man's life? What makes virtues possible?

    \section{Individual Rights as Absolutes}

        The basic principle of politics is the principle of individual rights. Inidividual rights are the means of subordinating society to moral law.

        \begin{definition}[Right]
            A moral or legal entitlement to have or obtain something or to act in a certain way.
        \end{definition}

        If your society is to be moral (and therefore practical), it declares, you must begin by recognizing the moral requirements of man in a social context; i.e., you must define the sphere of sovereignty mandated for every individual by the laws of morality.

        The fundamental right is the right to life. Its major derivatives are the right to liberty, property, and the pursuit of happiness.

        The right to life means the right to sustain and protect one's life. It means the right to take all the actions required by the nature of a rational being for the preservation of his life. To sustain his life, man needs a method of survival—he must use his rational faculty to gain knowledge and choose values, then act to achieve his values. The right to liberty is the right to this method; it is the right to think and choose, then to act in accordance with one's judgment. To sustain his life, man needs to create the material means of his survival. The right to property is the right to this process; in Ayn Rand's definition, it is "the right to gain, to keep, to use and to dispose of material values." To sustain his life, man needs to be governed by a certain motivehis purpose must be his own welfare. The right to the pursuit of happiness is the right to this motive; it is the right to live for one's own sake and fulfillment.

        Since man is an integrated being of mind and body, every right entails every other; none is definable or possible apart from the rest.

        Man is a certain kind of living organism—which leads to his need of morality and to man s life being the moral standard—which leads to the right to act by the guidance of this standard, i.e., the right to life. Reason is man's basic means of survival—which leads to rationality being the primary virtue—which leads to the right to act according to one s judgment, i.e., the right to liberty. Unlike animals, man- does not survive by adjusting to the given—which leads to productiveness being a cardinal virtue—which leads to the right to keep, use, and dispose of the things one has produced, i.e., the right to property. Reason is an attribute of the indi- vidual, one that demands, as a condition of its function, un- breached allegiance to reality—which leads to the ethics of egoism—which leads to the right to the pursuit of happiness.

        All rights rest on the fact that man's life is the moral stan- dard. Rights are rights to the kinds of actions necessary for the preservation of human life.

        All rights rest on the ethics of egoism. Rights are an in- dividual's selfish possessions—his title to his life, his liberty, his property, the pursuit of his own happiness.

        By its nature, the concept of a "right" pertains, in Ayn Rand's words, "only to action—specifically, to freedom of ac- tion. It means freedom from physical compulsion, coercion or interference by other men."
        
        A man's rights impose no duties on others, but only a negative obligation: oth- ers may not properly violate his rights.

        "Individual rights," in short, is a redundancy, albeit a neces- sary one in today's intellectual chaos. Only the individual has rights.

        The rights of man, Ayn Rand holds, can be violated by one means only: by the initiation of physical force (including its indirect forms, such as fraud).

        Metaphysically, the individual is sovereign (he is a being of self-made soul). Ethically, he is obliged to live as a sovereign (as an independent egoist). Politically, therefore, he must be able to act as a sovereign.

    \section{Government as an Agency to Protect Rights}

    \section{Statism as the Politics of Unreason} %---1,2,3
\chapter{Capitalism}

    \section{Capitalism as the Only Moral System}

    \section{Capitalism as the System of Objectivity}

    \section{Opposition to Capitalism as Dependent on Bad Epistemology} %---1,2,3
\part{Aesthetics}
\label{part:aesthetics}

\chapter{Art}

    \section{Art as a Concretization of Metaphysics}

    \section{Romantic Literature as Illustrating the Role of Philosophy in Art}

    \section{Esthetic Value as Objective} %---1,2,3

\backmatter

\bibliography{references}

\end{document}