\part{Ethics}
\label{part:ethics}

\chapter{Man}

    There is no question more crucial to man than the question: what is man? What kind of being is he? What are his essential attributes?

    A philosophical inquiry into man is not part of the special sciences: it is a study of man's metaphysical nature.

    In this inquiry, one is not concerned to discover what is right for man or wrong, desirable or undesirable, good or evil. The concern here is a purely factual question: what is the essence of human nature?

    Ultil you know what you are you cannot know what you ought to do. The root of all evaluative sciences is the nature of man.

    \section{Man as Conditional and Goal-Directed}

        Man is a living being, which means that he can die. His existence is not given: it is conditional (Definition \ref{def:conditional}).

            \begin{definition}[Conditional]
            \label{def:conditional}
                Subject to one or more conditions or requirements being met; made or granted on certain terms.
            \end{definition}

        Man's existence is subject to requirements. It requires a specific course of action, which itself requires effort. The actions of man (his efforts) are goal-directed (Definition \ref{def:goal}) if he is to live.

            \begin{definition}[Goal]
            \label{def:goal}
                The object of a person's ambition or effort; an aim or desired result.
            \end{definition}

    \section{Reason as Man's Basic Means of Survival}

            \begin{theorem}
                Reason is man's basic means of survival.
            \end{theorem}

            \begin{proof}
                Man, if he is to survive, needs to produce what his survival requires. In order to produce objects in reality he needs to deal with reality. For man's actions to be successful in dealing with reality, he needs to know reality. The method of knowing reality is reason. Reason, therefore, is man's basic tool of survival.
            \end{proof}


    \section{Reason as an Attribute of the Individual}

        Reason is an attribute of the individual, There is no such thing as a collective mind or brain. Thought is a process that must be initiated and directed at each step by the choice of one man, the thinker. Only an individual qua individual can perceive, abstract, define, connect.