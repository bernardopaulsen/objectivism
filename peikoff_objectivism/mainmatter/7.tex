\chapter{The Good}

    Ethis is the branch of philisophy that deals with moral principles (Defintion \ref{def:ethics}). Moral principles are to guide man's choices and actions - which determine the purpose and course of his life. Man needs a moral code because his life requires a specific course of action and, being a conceptual entity, he cannot follow this course except by the guidance of concepts.

    There are two key questions aswered by ethics:
    \begin{itemize}
        \item Fow what a man should live?
        \item By what fundamental principle should he act in order to achieve this end?
    \end{itemize}

    These quetions determine the ultimate value and the primary virtue. Our answers, at the end of this chapter, will be the following: the ultimate value is life (oneself's), and the primary virtue is rationality.

    The main problem of ethics, as a brach of phiolosophy that follows metaphysics and epistemology is: how can knowledge about what is bring to the knowledge of what ought to be? We will hold in this chapter that facts do lead to values.
    
    Ehics can be validated objectively, it is a science (it can be discovered with cognition) - and a human necessity.

    \section{``Life'' as the Essential Root of ``Value''}

        First, lets define ``value'' (Definition \ref{def:value}) and consequently define ``important'' (Definition \ref{def:important}).

            \begin{definition}[Value]
            \label{def:value}
                A person's principles or standards of behavior; one's judgment of what is important in life.   
            \end{definition}

            \begin{definition}[Important]
            \label{def:important}
                Of great significance or value; likely to have a profound effect on success, survival, or well-being.
            \end{definition}

        Now we are able to prove that life is the ultimate value (Theorem \ref{the:ultimate}), and, accordingly, that remaining alive is the goal of values and of all proper action.

            \begin{theorem}
            \label{the:ultimate}
                ``Life'' is the ultimate value.
            \end{theorem}

            \begin{proof}
                A value is something important - something that helps survival (the maintence of life). Survival - life - is the ultimate value. The most important thing in life is life itself.
            \end{proof}

        Value pressuposes goal-directed action (behaviour). An object is outside the field of value if action in relation to it is inapplicable or ineffectual. Living organisms, therefore, are the entities that make value possible. They, nevertheless, do not exist in order to pursue values: they pursue values in order to exist.

        Only self-preservation can be an ultimate goal, which serves no end beyond itself. Philosophically speaking, the essence of self-preservation is: accepting the realm of reality.

    \section{Man's Life as the Standard of Moral Value}

       Man has a nature, he must follow a specific course of action if he is to survive. But man is not born knowing what that course is, nor does such knowledge well up in him effortlessly. To know this course is the purpose of morality (Definition \ref{def:morality}).

            \begin{definition}[Morality]
            \label{def:morality}
                \begin{enumerate}
                    \item Principles concerning the distinction between right and wrong or good and bad behavior,
                    \item a particular system of values and principles of conduct, especially one held by a specified person or society.
                \end{enumerate}
            \end{definition}

            \begin{definition}[Principle]
                \label{def:principle}
                    A fundamental truth or proposition that serves as the foundation for a system of belief or behavior or for a chain of reasoning.
            \end{definition}

        Man (and only man) needs morality in order to survive. Moral laws are principles that define how to nourish and sustain human life. It is the science of human self-preservation. Man needs a moral code because his life requires a specific course of action and, being a conceptual entity, he cannot follow this course except by the guidance of concepts.
        
        No action an organism takes is irrelevant to its existence, therefore, morality applies to every action. For a rational being, principled action is the only effective kind of action.

        A code of values based on reason and proper to man must hold life as the ultimate value. All that which is proper to the life of a rational being is the good; all that which destroys it is the evil.

    \section{Rationality as the Primary Virtue}

        What are the principles of human survival? What objects must man hold as values if he is to preserve his life, and what virtues must he practice in order to achieve them?

        The faculty of reason is man's basic tool of survival. The primary choice is to exercise this faculty or not. If life is the standard, therefore, the basic moral principle is obvious. It tells us the proper evaluation of reason.

        To live, man must hold three things as the supreme and ruling values of his life: Reason—Purpose—Self-esteem. Reason, as his only tool of knowledge—Purpose, as his choice of the happiness which that tool must proceed to achieve—Self-esteem, as his inviolate certainty that his mind is competent to think and his person is worthy of happiness, which means: is worthy of living. These three values imply and require all of man's virtues...

        Epistemology tells us reason is valid. Ethics tells us reason is a value - is important. If one choses to live, one must hold reason as a value.

            \begin{definition}[Virtue]
            \label{def:virtue}
                The action by which one gains and keeps a value.
            \end{definition}

        "Rationality," according to Ayn Rand, is "the recognition and acceptance of reason as one's only source of knowledge, one's only judge of values and one's only guide to action."

        This means the application of reason to every aspect of one's life and concerns. It means choosing and validating one's opinions, one's decisions, one's work, one's love, in accor- dance with the normal requirements of a cognitive process, the requirements of logic, objectivity, integration. Put nega- tively, the virtue means never placing any consideration above one's perception of reality. 

        By the same token, there is only one primary vice, which is the root of all other human evils: irrationality. This is the deliberate suspension of consciousness, the refusal to see, to think, to know.

        Now let me con- sider certain aspects of rationality in greater detail.

        To begin with, one cannot follow reason unless one ex- ercises it. Rationality demands continual mental activity. Rationality requires the systematic use of one's intelligence.

        The point is not that one must become a genius or even an intellectual. Contrary to a widespread fallacy, reason is a faculty of human beings, not of "supermen." The moral point here is always to grow mentally, to increase one's knowledge and expand the power of one's consciousness to the extent one can, whatever one's profession or the degree of one's intelligence.

        The men of virtue are the men who choose to practice and welcome this kind of struggle on principle, as a lifelong commitment.

        Just as every idea has a relationship to one's other ideas, and none can be accepted until it is seen to be an element of a single cognitive whole; so every fact has a relationship to other facts, and none can be evaded without tearing apart and destroying that kind of whole.

        The existential side of rationality is the policy of acting in accordance with one's rational conclusions. There is no point in using one's mind if the knowledge one gains thereby is not one's guide in action.

        This aspect of rationality subsumes several obligations.29 It requires that one choose not only his abstract values but also his specific goals by a process of rational thought. It requires that one know what his motives are. It requires that one choose the means to his ends by reference to explicitly defined principles, both moral and scientific. And it requires that one then enact the means, accepting the law of causality in full.

        In epistemology, we concluded that emotions are not tools of cognition. The corollary in ethics is that they are not guides to action.

        The proper approach in this issue is not reason versus emotion, but reason first and then emotion. This approach, as we have seen, leads to the harmony of reason and emotion, which is the normal state of a rational man. 

    \section{The Individual as the Propor Beneficiary of his Own Moral Action}

        Each individual must choose his values and actions by the standard of man's life—in order to achieve the purpose of maintaining and enjoying his own life.

        Thus Objectivism advocatesegoism—thepursuit of self-interest—the policy of selfishness.

        It simply states: whatever man's proper self-interest consists of, that is what each individual should seek to achieve.

        In the Objectivist view, the validation of egoism consists in showing that it is a corollary of man's life as the moral standard.

        "Only the alternative of life vs. death," I said earlier, "creates the context for value-oriented action. . . ." And "only self-preservation," I said, "can be an ultimate goal.

        "Egoistic," in the Objectivist view, means self-sustaining by an act of choice and as a matter of principle.

        The wider principle demanding such egoism is the fact that survival requires an all-encompassing course of action. If an action of his is not for his life, then, as we have seen, it is against his life

        [...]
        
    \section{Values as Objective}

        Values, like concepts, are not intrinsic or subjective, but objective. Values are not intrinsic features of reality, they require a valuer - the good is not good in itself, it is good to man and for the sake of he's life. Values are neither subjective, they require the observation of reality - the realm of facts is what creates the need to choose a certain goal.

        Moral value does not pertain to reality alone or to consciousness alone, it arises because a certain kink of entity - a living, volitional, conceptual organism - sustains a certain realtionship to the external world. The good is an aspect of reality in relationship to man. The good designates facts - the requirements of survival - as identified conceptually.

        Morality is  a must - if; it is the price of the choice to live. That choice itself, is not a moral choice, it preceded morality; it is the decision of consciousness that uderlies the need of morality.

        Suicide, nevertheless, is sometimes justified.




        