\part{Politics}
\label{part:politics}

\chapter{Government}

    Politics, like ethics, is a normative branch of philosophy. Politics defines the principles of a proper social system, including the proper functions of government. Politics rests on ethics (and thus on metaphysics and spistemoloy); it is an application of ethics to social questions.

    What type of society conforms to or reflects the principles of morality? - this is the question asked by philosophicas politics. In Objectivism, the question is: what type of society conforms to the requirements of man's life? What makes virtues possible?

    \section{Individual Rights as Absolutes}

        The basic principle of politics is the principle of individual rights. Inidividual rights are the means of subordinating society to moral law.

        \begin{definition}[Right]
            A moral or legal entitlement to have or obtain something or to act in a certain way.
        \end{definition}

        If your society is to be moral (and therefore practical), it declares, you must begin by recognizing the moral requirements of man in a social context; i.e., you must define the sphere of sovereignty mandated for every individual by the laws of morality.

        The fundamental right is the right to life. Its major derivatives are the right to liberty, property, and the pursuit of happiness.

        The right to life means the right to sustain and protect one's life. It means the right to take all the actions required by the nature of a rational being for the preservation of his life. To sustain his life, man needs a method of survival—he must use his rational faculty to gain knowledge and choose values, then act to achieve his values. The right to liberty is the right to this method; it is the right to think and choose, then to act in accordance with one's judgment. To sustain his life, man needs to create the material means of his survival. The right to property is the right to this process; in Ayn Rand's definition, it is "the right to gain, to keep, to use and to dispose of material values." To sustain his life, man needs to be governed by a certain motivehis purpose must be his own welfare. The right to the pursuit of happiness is the right to this motive; it is the right to live for one's own sake and fulfillment.

        Since man is an integrated being of mind and body, every right entails every other; none is definable or possible apart from the rest.

        Man is a certain kind of living organism—which leads to his need of morality and to man s life being the moral standard—which leads to the right to act by the guidance of this standard, i.e., the right to life. Reason is man's basic means of survival—which leads to rationality being the primary virtue—which leads to the right to act according to one s judgment, i.e., the right to liberty. Unlike animals, man- does not survive by adjusting to the given—which leads to productiveness being a cardinal virtue—which leads to the right to keep, use, and dispose of the things one has produced, i.e., the right to property. Reason is an attribute of the indi- vidual, one that demands, as a condition of its function, un- breached allegiance to reality—which leads to the ethics of egoism—which leads to the right to the pursuit of happiness.

        All rights rest on the fact that man's life is the moral stan- dard. Rights are rights to the kinds of actions necessary for the preservation of human life.

        All rights rest on the ethics of egoism. Rights are an in- dividual's selfish possessions—his title to his life, his liberty, his property, the pursuit of his own happiness.

        By its nature, the concept of a "right" pertains, in Ayn Rand's words, "only to action—specifically, to freedom of ac- tion. It means freedom from physical compulsion, coercion or interference by other men."
        
        A man's rights impose no duties on others, but only a negative obligation: oth- ers may not properly violate his rights.

        "Individual rights," in short, is a redundancy, albeit a neces- sary one in today's intellectual chaos. Only the individual has rights.

        The rights of man, Ayn Rand holds, can be violated by one means only: by the initiation of physical force (including its indirect forms, such as fraud).

        Metaphysically, the individual is sovereign (he is a being of self-made soul). Ethically, he is obliged to live as a sovereign (as an independent egoist). Politically, therefore, he must be able to act as a sovereign.

    \section{Government as an Agency to Protect Rights}

    \section{Statism as the Politics of Unreason}