\chapter{Happiness}

    The existential reward of virtue is life, and the emotional reward is happiness. Happiness is the \textit{only} moral purpuse of man's life.

    \section{Virtue as Practical}

        \begin{definition}[Practical]
        \label{def:practical}
            \begin{enumerate}
                \item Of or concerned with the actual doing or use of something rather than with theory and ideas.
                \item (Of an idea, plan, or method) likely to succeed or be effective in real circumstances; feasible.
            \end{enumerate}
        \end{definition}

        The moral man's concept of the good - life - is his fundamental standard of practicality.

    \section{Happiness as the Normal Condition of Man}

        There are two oposite emotions - the first follows the achievement of value, the other from its loss - they are joy and suffering (Definitions \ref{def:joy} and \ref{def:suffering}). Happiness (Definition \ref{def:happiness}) follows from joy.

        \begin{definition}[Joy]
        \label{def:joy}
            Emotion that follows from the achienvement of values.
        \end{definition}

        \begin{definition}[Suffering]
        \label{def:suffering}
            Emotion that follows from the loss of values.
        \end{definition}

        \begin{definition}[Happiness]
        \label{def:happiness}
            A state of noncontradictory joy.
        \end{definition}

        To feel happiness, it is necessary to achieve noncontradictory - rational - values. Therefore the moral, the practical and the happy are the same. To feel happiness is to feel a metahphysical pleasure: the pleasure of being in the right relationship to existence. Happiness is the purpose of ethics, it is an end in itself. A rational man must fight for and expect happiness.

        While joy and suffering follow from virtue, which pressuposes evaluation, from evaluation itself love and fear (Definitions \ref{def:love} and \ref{def:fear}) follow.

        \begin{definition}[Love]
        \label{def:love}
            Emotion that follows from the appreciation of values.
        \end{definition}

        \begin{definition}[Fear]
        \label{def:fear}
            Emotion that follows from the depreciation of disvalues.
        \end{definition}

    \section{Sex as Metaphysical}

        Here we shall consider sex as in the life of a rational man.

        To celebrate something is to focus on it. Sex is a celebration of love. Love itself is the appreciation of values. Since the value of a man is his virtue and the latter is his volition, which is who he truly is, to celebrate self love is to celebrate oneself. To celebrate love towards the other is also to celebrate who the other truly is.

        