\chapter{Reason}
\label{cha:reason}

    The whole of our phisolophy amounts to ``follow reason''. ``Reason'' (Definition \ref{def:reason}), nevertheless, is a higher-level concept, and to grasp its meaning one must first grasp its hierarchical roots. These are what we have been discussing about in the chapters above.

        \begin{definition}[Reason]
        \label{def:reason}
            Method of cognition that proceeds in accordance with facts, which are established, directly or indirectly, by observation.
            
            The faculty that:
            \begin{itemize}
                \item identifies and integrates the material provided by man's senses;
                \item enables man to discover the nature of existents — by virtue of its power to condense sensory information in accordance with the requirements of an objective mode of cognition;
                \item organizes perceptual units in conceptual terms by following the principles of logic.
            \end{itemize}
        \end{definition}

        \begin{remark}
            The latter formulation highlights the three elements essential to the faculty: its data, percepts; its form, concepts; its method, logic.
        \end{remark}

    Reason is the existence-oriented faculty. Accepting reason is accepting reality.

    \section{Emotions as a Product of Ideas}
    
        First, let's distinguish emotion from sensation. Sensation follows Definition \ref{def:sensation}. It is automatic, independent of ideas. An emotion (Definition \ref{def:emotion}), on the other hand, is dependent on ideas.

            \begin{definition}[Emotion]
            \label{def:emotion}
                A state of consciousness which is a response to an object one perceives (or imagines).
            \end{definition}

        An emotion arises only if two necessary conditions are met:

        \begin{enumerate}
            \item one must identify the object perceived (or imagined),
            \item one must evaluete the object.
        \end{enumerate}
    
        An individual cannot have an emotional response to and object without identity, that means nothing. A nothing cannot be evaluated. He also cannot have an emotion driven by something de does not evaluate. Seomthing not evaluated may have identity, but does not have any meaning.
    
        There are four steps in the generation of an emotion:

        \begin{enumerate}
            \item perception (or imagination),
            \item identification,
            \item evaluation,
            \item reponse.
        \end{enumerate}
    
        Only the first and last step are necessarily conscious. The other two may occur without the need of conscious awareness, as once an individual has formed value-judgments, he automatizes them.
        
        Value-judgments are formed by thoughts, and ultimately on a philosophic view of oneself and others; of man, life and the universe. Such a view conditions all one's emotions. These views may be held implicitly of implicitly.
        
        What makes emotions incomprehensible to many people is the fact that their views are not only implicit, but contracditory. This leads to the appearance of a conflict between thought and feelings.

    \section{Reason as Man's Only Means of Knowledge}
    
            \begin{theorem}
                Reason is man's only means of knowledge
            \end{theorem}

            \begin{proof}
                There are two candidates for means of knowledge: reason and emotion.

                Reason, following Definition \ref{def:reason} is a method of cognition that proceeds in accordance with facts, which are established, directly or indirectly, by observation. Emotion, on the other hand, following Definition \ref{def:emotion}, is a state of consciousness which is a response to an object one perceives (or imagines).

                Emotion is the result of past conclusions. It, consequently, cannot be trust as a means to knowledge, as our past conclusios may be wrong. Reason is exactly what identifies the true and false whithin our thougths, as it compares them to the facts.
            \end{proof}
            
        Reason is a faculty of awareness; its function is to perceive that which exists by organizing observational data. And reason is a volitional faculty; it has the power to direct its own actions and check its conclusions, the power to maintain a certain relationship to the facts of reality. Emotion, by contrast, is a faculty not of perception, but of reaction to one's perceptions. This kind of faculty has no power of observation and no volition; it has no means of independent access to reality, no means to guide its own course, and no capacity to monitor its own relationship to facts.
        
        Emotions are automatic consequences of a mind's past conclusions. Feeling follows obediently. It has no power to question its course or to check its roots against reality. Only man's volitional, existence-oriented faculty has such power.
        
        Now, through a study of man's means of consciousness, this earlier discussion has been confirmed and completed. Metaphysics and epistemology unite. They unite in declaring that "emotions are not tools of cognition.
        
        The conclusion is clear: there is no alternative or supplement to reason as a means of knowledge. If one attempts to give emotions such a role, then he has ceased to engage in the activity of cognition.
        
        If an individual experiences a clash between feeling and thought, he should not ignore his feelings. He should identify the ideas at their base (which may be a time-consuming process); then compare these ideas to his conscious conclusions, weighing the conflicts objectively; then amend his viewpoint accordingly, disavowing the ideas he judges to be false
        
        The above indicates the pattern of the proper relationship between reason and emotion in a man's life: reason first, emotion as a consequence.

    \section{The Arbitrary as Neither True or False}
    
        An arbitrary claim is one for which there is no evidence, either perceptual or conceptual. an arbitrary claim is automatically invalidated. The rational response to such a claim is to dismiss it, without discussion, consideration, or argument
        
        An arbitrary statement has no relation to man's means of knowledge. Since the statement is detached from the realm of evidence, no process of logic can assess it
        
        Since it is affirmed in a void, cut off from any context, no integration to the rest of man's knowledge is applicable; previous knowledge is irrelevant to it. Since it has no place in a hierarchy, no reduction is possible, and thus no observations are relevant.
        
        If an idea is cut loose from any means of cognition, there is no way of bringing it into relationship with reality
        
        An arbitrary idea must be given the exact treatment its nature demands. One must treat it as though nothing had been said. The reason is that, cognitively speaking, nothing has been said. One cannot allow into the realm of cognition something that repudiates every rule of that realm.
        
        The true is identified by reference to a body of evidence; it is pronounced "true" because it can be integrated without contradiction into a total context. The false is identified by the same means; it is pronounced "false" because it contradicts the evidence and/or some aspect of the wider context. The arbitrary, however, has no relation to evidence or context; neither term, therefore—"true" or "false"—can be applied to it.
        
        The onus of proof rule states the following. If a person asserts that a certain entity exists (such as God, gremlins, a disembodied soul), he is required to adduce evidence supporting his claim. If he does so, one must either accept his conclusion, or disqualify his evidence by showing that he has misinterpreted certain data. But if he offers no supporting evidence, one must dismiss his claim without argumentation, because in this situation argument would be futile. It is impossible to "prove a negative," meaning by the term: prove the nonexistence of an entity for which there is no evidence
        
        a nonexistent is nothing; it is not a constituent of reality, and it has no effects. If gremlins, for instance, do not exist, then they are nothing and have no consequences. In such a case, to say: "Prove that there are no gremlins/' is to say: "Point out the facts of reality that follow from the nonexistence of gremlins." But there are no such facts. Nothing follows from nothing.

    \section{Certainty as Contextual}
    
        Human knowledge is limited. Logical processing of an idea within a specific context of knowledge is necessary and sufficient to establish the idea's truth.
        
        Consciousness has identity, and epistemology is based on the recognition of this fact. Epistemology investigates the question: what rules must be followed by a human consciousness if it is to perceive reality correctly? Nothing inherent in human consciousness, therefore, can be used to undermine it

        If a fact is inherent in human consciousness, then that fact is not an obstacle to cognition, but a precondition of it—and one which implies a corresponding epistemological obligation

        Man is a being of limited knowledge—and he must, there fore, identify the cognitive context of his conclusions.

        The im plicit or explicit preamble to his conclusion must be: "On the basis of the available evidence, i.e., within the context of the factors so far discovered, the following is the proper conclu sion to draw.

        If a man follows this policy, he will find that his knowl edge at one stage is not contradicted by later discoveries.

        The advanced conclusions augment and enhance his earlier knowledge; they do not clash with or annul it.

        The principle here is evident: since a later discovery rests hierarchically on earlier knowledge, it cannot contradict its own base.
        
        Although the researchers cannot claim their discovery as an out-of-context absolute, they must treat it as a contextual absolute (i.e., as an immutable truth within the specified con text).

        A man does not know everything, but he does know what he knows.
        
        This is why there can be no such thing as "some evi dence" in favor of an entity transcending nature and logic The term "evidence" in this context would be a stolen con cept. Since nothing can ever qualify as a "proof" of such an entity, there is no way to identify any data as being a "part proof" of it, either.