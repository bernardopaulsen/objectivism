\chapter{Virtue}

    Above we argued rationality is the primary virtue. Now, we turn to the derivative virtues of rationality.

    \section{Independence as a Primary Orientation to Reality, not to Other Men}

        \begin{definition}[Independence]
        \label{def:independence}
            One's accpetance of the responsibility of forming one's own judgments and of living by the work of one's own mind.
        \end{definition}

        Rationality is the primary virtue, and reason is a faculty of the individual. Therefore, in order to be rational, one must think independently (a pleonasm). This means not accepting thoughts or jugdments of others, but to form one's own. This does not mean never thinking like other people or accepting ideas created by other, but it does mean that this acceptance must be rational, the thoughts must be integrated by an act of reason - thinking.

    \section{Integrity as Loyalty to Rational Principles}

        \begin{definition}[Integrity]
        \label{def:integrity}
            Loyalty in action to one's convictions and values.
        \end{definition}

        Independence is forming one's own ideas. Integrety is following one's own ideas, acting rationally. Reality, to be mastered, must be known, but it also requires that man acts on his knowledge. There is no use for knowledge if it is not to be acted upon.

    \section{Honesty as the Rejection of Unreality}

        

    \section{Justice as Rationality in the Evaluation of Men}

        

    \section{Productiveness as the Adjustment of Nature to Man}

        \begin{definition}[Productiveness]
        \label{def:productiveness}
            The process of creating material values.
        \end{definition}

        Productiveness is a necessity of human survival. Just as there is not such a thing as ``too much'' knowledge, there is not such a thing as ``too much'' wealth. Intelectually, every discovery enhances human life; existentially, every material achievement also enhances human life.

        Productiveness is the creation of \textit{material} values, not because knowledge is important, but because it is important exclactly because it is useful in creating material values.

    \section{Pride as Moral Ambitiousness}

        \begin{definition}[Pride]
            A feeling of deep pleasure or satisfaction derived from one's own achievements, the achievements of those with whom one is closely associated, or from qualities or possessions that are widely admired.
        \end{definition}

        \begin{definition}[Self-esteem]
            Confidence in one's own worth or abilities; self-respect.
        \end{definition}

        Pride and self-esteem are results of virtue, of moral action.


    \section{The Initiation of Physical Force as Evil}

        Initiation of physical force is the destruction of rationality, and therefore of every other virtue and of every value. To force a man to do something is to deny him the possibility of acting on reason, thus turning any virtuous action (value achievement) impossible.
        
        A (rational) victim of force still thinks what he thinks, as force cannot change the content of one's mind. Thus the victim of force acts \textit{against} his judgement, and his cognition becomes useless. As a rational mind acts virtuously, to act against its judgement is to act perversely, to act against one's own life. Initiating force, when it does not kill the victim, kills his capacity to live.

        Coercion places the individual in an impossible metaphysical condition. If he does not act in accordance with his conlusions, he is doomed by reality. If he does, he is doomed by the forcer.

        Finally, one important thing to notice is that only initiation - not retaliation - of physical force is evil. While initiation of physical force destroys rationality, retaliation protects it. It cannot creates values, however - it can only protect them.